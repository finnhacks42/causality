\documentclass{article}

\usepackage{times}
\usepackage{graphicx} % more modern
\usepackage{subfigure} 
\usepackage{natbib}
\usepackage{algorithm}
\usepackage{algorithmic}

\newif\ifsup\supfalse



\usepackage{hyperref}
\newcommand{\theHalgorithm}{\arabic{algorithm}}
\usepackage{tikz}
\usepackage{dsfont}
\usepackage{amsmath}
\usepackage{amssymb}
\usepackage{amsthm}



%\setlength{\marginparwidth}{10ex}%added by Yifan so that the todos can fit in the margin
%\usepackage[obeyFinal]{todonotes}
\usepackage[disable]{todonotes}

\newcommand{\tinytodo}[2][]{\todo[size=\tiny]{#2}}
%\newcommand{\tinytodo}[2][]{\todo[size=\tiny, #1]{\begin{spacing}{1.0}#2\end{spacing}}}
\newcommand{\todot}[2][]{\tinytodo[color=blue!20, #1]{T: #2}} % Tor
\newcommand{\todof}[2][]{\tinytodo[color=red!20, #1]{F:\@#2}} % Finn
\newcommand{\todom}[2][]{\tinytodo[color=green!20, #1]{M:\@#2}} % Mark


\graphicspath{ {figures/} }
\usetikzlibrary{arrows,positioning} 
\tikzset{
    %Define standard arrow tip
    >=stealth',
    %Define style for boxes
    observed/.style={
           circle,
           rounded corners,
           draw=black, thick,
           minimum width=2.5em,
           minimum height=2.5em,
           font=\footnotesize,
           text centered,
           fill=blue!20!white},
     latent/.style={
           circle,
           rounded corners,
           draw=black, thick, dashed,
           minimum width=.5em,
           minimum height=.5em,
           font=\footnotesize,
           text centered,
           fill=black!10!white
           },
     empty/.style={
           circle,
           rounded corners,
           minimum width=.5em,
           minimum height=.5em,
           font=\footnotesize,
           text centered,
           },
    % Define arrow style
    pil/.style={
           o->,
           thick,
           shorten <=2pt,
           shorten >=2pt,},
    sh/.style={ shade, shading=axis, left color=red, right color=green,
    shading angle=45 }  
}

\newcommand{\defined}{\vcentcolon =}
\newcommand{\rdefined}{=\vcentcolon}
\newcommand{\E}[1]{\mathbb E\left[#1\right]}
\newcommand{\R}{\mathbb R}
\newcommand{\Var}{\operatorname{Var}}
\newcommand{\calF}{\mathcal F}
\newcommand{\sr}[1]{\stackrel{#1}}
\newcommand{\set}[1]{\left\{#1\right\}}
\newcommand{\ind}[1]{\mathds{1}\!\!\set{#1}}
\newcommand{\argmax}{\operatornamewithlimits{arg\,max}}
\newcommand{\argmin}{\operatornamewithlimits{arg\,min}}
\newcommand{\floor}[1]{\left \lfloor {#1} \right\rfloor}
\newcommand{\ceil}[1]{\left \lceil {#1} \right\rceil}
\newcommand{\eqn}[1]{\begin{align}#1\end{align}}
\newcommand{\eq}[1]{\begin{align*}#1\end{align*}}
\newcommand{\Ber}{\operatorname{Bernoulli}}
\renewcommand{\P}[1]{\operatorname{P}\left\{#1\right\}}
\newcommand{\Pri}[1]{\operatorname{P}_i\left\{#1\right\}}
\newcommand{\Prz}[1]{\operatorname{P}_0\left\{#1\right\}}
\newcommand{\bigo}[1]{\mathcal{O}\left( #1 \right)}
\newcommand{\bigotilde}[1]{\tilde{\mathcal{O}}\left( #1 \right)}
\newcommand{\bigtheta}[1]{\Theta\left( #1 \right)}
\newcommand{\bigthetatilde}[1]{\tilde{\Theta}\left( #1 \right)}
\newcommand{\bigomega}[1]{\Omega\left( #1 \right)}
\newcommand{\KL}{\operatorname{KL}}
\newcommand{\simpleregret}{R_T}
\newcommand{\Pij}[1]{\operatorname{P}_{ij}\!\left\{#1\right\}}
\newcommand{\Pkl}[1]{\operatorname{P}_{kl}\!\left\{#1\right\}}
\newcommand{\Q}[1]{\operatorname{Q}\left\{#1\right\}}
\newcommand{\EE}{\mathbb E}
\newcommand{\Pn}[2]{\operatorname{P}_{#1}\left\{#2\right\}}
\newcommand{\parents}[1]{\operatorname{Pa}_{#1}}
\newcommand{\actions}{\mathcal{A}}
\newcommand{\calA}{\mathcal A}

\newcommand{\etc}{\textit{etc}}
\newcommand{\ie}{\textit{i.e.}}
\newcommand{\eg}{\textit{e.g.}}
\newcommand{\latent}{\mathbf{L}}
\newcommand{\observed}{\mathbf{O}}
\newcommand{\variables}{\mathbf{V}}

\theoremstyle{plain}
\newtheorem{theorem}{Theorem}
\newtheorem{proposition}[theorem]{Proposition}
\newtheorem{lemma}[theorem]{Lemma}
\newtheorem{corollary}[theorem]{Corollary}
\theoremstyle{definition}
\newtheorem{definition}[theorem]{Definition}
\newtheorem{assumption}[theorem]{Assumption}
\newtheorem{remark}[theorem]{Remark}
\newtheorem{example}[theorem]{Example}
\let\temp\epsilon
\let\epsilon\varepsilon


% The \icmltitle you define below is probably too long as a header.
% Therefore, a short form for the running title is supplied here:


\begin{document} 


%%%%%%%%%%%%%%%%%%%%%%%%%%%%%%%%%%%%%%%%%%%%%%%%%
% INTRODUCTION
%%%%%%%%%%%%%%%%%%%%%%%%%%%%%%%%%%%%%%%%%%%%%%%%%
\section*{Identifiability}

We have three questions relating to the general idea of identifiability in causal systems. For all cases, assume we are working with a causal directed acyclic graph. Initially, lets only consider deterministic interventions on single variables in a DAG (not as restrictive as it seems since non-deterministic/multiple interventions can still be modelled by adding nodes to the graph). More formally, we have a causal directed acyclic graph $G$ over variables $\variables = \observed \cup \latent$ where $\observed$ are observable and $\latent$ are latent. Let the set of causal queries $\mathcal{Q}$ be $P(Y|do(X))$ for all possible pairings of variables $X$,$Y$ in $\observed$.

\begin{definition}[Graphical identifiability] a query $Q = P(Y|do(X))$ is identifiable iff we can write an expression for it in terms of conditional probabilities of variables in $\observed$. 
\end{definition}

If a query is graphically identifiable then we can asymptotically obtain the interventional distribution of $Y$ given $do(X=x)$ for any value of $x$ in any probability distribution $P$ compatible with $G$ (with some assumptions on probabilities not being 0). Any desired metric (eg difference in mean, KL divergence) can be applied to the pre and post-interventional distributions to obtain a summary of the \textit{causal effect}. 

\subsection*{Partial identifiability}

The graphical identifiability criterion is boolean. There are two directions we can come at trying to smooth this boundary.

\begin{enumerate}
\item Quantify the difficulty in identifying causal effects for problems that are graphically identifiable. What is the variance in the estimators at finite time? What is the minimal change in the graph that makes it non-identifiable? 

\item Quantify how well we can identify causal effects for problems that are not graphically identifiable. For example, how tight are the bounds we can get, what is the minimal change in the graph that would make the problem identifiable (eg observe nodes in $\mathcal{U}$ or delete edges),what minimal assumptions must we make about functional relationships between variables to get a good estimate. 
\end{enumerate}

\subsection*{Causal Privacy}

Closely related to the question of partial identifiability. What changes can we make that take a problem from identifiable to non-identifiable. 

\subsection*{Characterizing graphical identifiability}

There is an algorithm that determines if a given query in a causal graph is identifiable or not. \ref{} There are also a number of simple graphical criteria that can be used to determine if a graph is or is not identifiable (but no single simple necessary and sufficient criteria). These are:

\begin{itemize}
\item The backdoor criteria (sufficient)
\item The frontdoor criteria (sufficient)
\item (nesseesary)
\end{itemize}

In practise, the back-door criterion is used to estimate causal effects in the vast majority of applications. The front door criterion is also used occasionally. 

The question is - can we divide the space of causal queries into sets depending on how they can (or cannot) be identified and estimate the size of the each set? Are there interesting features of graphs that predict identifiability?

A given query can be identified in more than one way. We could try to determine all the ways in which a query could be identified or apply a hierarchy, for example: identifiable by backdoor, identifiable by frontdoor (but not backdoor, identifiable but not via backdoor or frontdoor, not identifiable. 

There are three key components to this problem. Identifying a space of graphs, traversing the space in an efficient way and algorithms to compute the quantities we are interested in. There are a very large number of possible causal graphs, even  if we restrict the number of nodes. There are also a number of ways of defining the space of graphs (for example all DAGs versus all MAGs). The existing Identify algorithm \ref{} determines if a query is identifiable or not - and returns an expression if it its. It does not enumerate all the expressions via which a query could be identified. There are algorithms that can determine if a query can be identified via a backdoor criterion \ref{}. No such algorithm that I know of for the front-door criterion. Ideally we want an algorithm that can traverse the space of graphs and efficiently utilize information from the previous (related query) to compute identifiability of the next query. 

Related questions - 
\begin{itemize}
\item Are the expressions that the Idenfiy algorithm outputs optimal? Ie - do backdoor criterion adjustments with fewer variables get emmitted in preference to backdoor adjustments with larger numbers of variables? 
\item If there are multiple approaches to identify a given query, how do we define which is optimal? What assumptions do we need to make about the data to define optimality? 
\item How does one efficiently compute causal effects from such expressions \ref{Richardson, etc}?
\end{itemize}

 





\end{document}
