
\newcommand{\calP}{\mathcal P}
\newcommand{\x}{\boldsymbol{x}}

We now consider single variable interventions on general causal graphs over variables $\boldsymbol{V}$. Let $Y \in \set{0,1}$ be the target variable and $\boldsymbol{X} = V/\set{Y \cup Desendants(Y)}$. 

If all the variables in $\boldsymbol{X}$ are observable, we can compute the distribution $P(Y,\boldsymbol{X}|do(X_i = j)$ for any intervention $i,j$ from data collected without intervening via the truncated product formula \cite{Pearl2000} 1.37.

\eq {
P(Y,X) = \prod_{k \neq i}\P{x_k|\calP_k(\x)}\P{Y|\calP_Y(\x)}\, \text{  $\forall \,\x$ consistent with $X_i = j$}  
} 

Where $\calP_i : \set{0,1}^N \to \set{0,1}^{N_i}$ is the
projection that selections the $N_i$ parents of variable $i$. 

We can then marginalize to obtain $P(Y|do(X_i = j))$

\eq {
P(Y|do(X_i = j)) =& \sum_{\x:x_i=j} \prod_{k \neq i}\P{x_k|\calP_k(\x)}\P{Y|\calP_Y(\x)}\\
 =& \sum_{\x:x_i=j} \prod_{k \neq i}\P{x_k|\calP_k(\x)}\P{Y|\x}\\
 =& \sum_{\x:x_i=j} \P{\boldsymbol{X}|do(X_i=j)}\P{Y|\x}
}

Let $r(x) = \P{Y = 1|X = x}$. Then

\eq {
\mu_{ij} = P(Y = 1|do(X_i = j)) = \sum_{\x:x_i=j} \P{\boldsymbol{X}|do(X_i=j)}r(x)
}


We wish to devise an estimator for $\mu_{ij}$ using $X$ and $Y$ without
intervening to set $X_i = j$. A natural choice is the importance weighted estimator

\eq{
Z_{ij} 
&= \frac{Y \ind{X_i = j}}{\P{X_i = j|\calP_i(X)}}\,. 
}

Checking:
\eq{
&\E{Z_{ij}} 
= \sum_x \frac{\P{X = x} r(x) \P{X = x | do(X_i = j)}}{\P{X = x}} \\
&= \P{Y = 1| do(X_i = j)} \,. 
}
The variance of $Z_{ij}$ may also be bounded
\eq{
&\Var[Z_{ij}]
\leq \E{Z_{ij}^2} \\
&\leq \sum_x \P{X = x} \left(\frac{\ind{X_i = j}}{\P{X_i = j|\calP_i(X) = \calP_i(x)}}\right)^2 \\
&= \E{\frac{1}{\P{X_i = j|\calP_i(X)}}} \\
&= V_{ij}\,.
}
In the general graph setting the dimension $m$ has a slightly different definition.
\eq{
m = \min\set{m : \sum_{ij} \ind{V_{ij} \leq 1/m} = m}\,.
}

\todof{I think the above should be}

\eq {
m = \min \set{m: \sum_{ij} \ind{V_{ij} \geq m} = m}\,.
}

That is, $m$ should be chosen such that the size of the set variances greater than $m$ is $m$. This ensures all the variances of the remaining arms are smaller than $m$. This definition would also seem equivalent to sorting the variances such that $V^{(0)} \geq V^{(1)} ... \geq V^{(2N)}$ and then defining $m = \min m: V^{(m)} \leq m$ which is in turn equivalent to the definition in terms of $q$ for parallel bandits. 

\begin{algorithm}[H]
\caption{General Algorithm}
\begin{algorithmic}
\STATE {\bf Input:} $N$, $\mathcal P$
\STATE Compute $m$:
\eq{
m = \min\set{m : \sum_{ij} \ind{V_{ij} \leq 1/m} = m}
}
\STATE Collect $T/2$ observational samples and estimate $\hat \mu_{ij}$:
\eq{
\hat \mu_{ij} = \sum_{t=1}^{T/2} \frac{Y_t \ind{X_{t,i} = j}}{\P{X_{t,i} = j | \mathcal P_i(X_t)}}
}
\FOR{$i \in \set{1,\ldots,m}$}
\STATE For each $i,j $ with $V_{ij} < 1/m$ take action $do(X_i = j)$ for $T/(2m)$ samples $Y^{(ij)}_1,\ldots,Y^{(ij)}_{T/2m}$
\STATE Update estimate:
\eq{
\eta &= \frac{V_{ij}}{V_{ij} + m} \\
\hat \mu_{ij} &= \eta \left(\frac{2m}{T} \sum_{t=1}^{T/(2m)} Y^{(ij)}_t\right) + (1 - \eta) \hat \mu_{ij}
}
\ENDFOR
\STATE Choose $do(X_i = j)$ for $ij = \argmax_{ij} \hat \mu_{ij}$
\end{algorithmic}
\end{algorithm}





