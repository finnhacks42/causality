In this section we develop and analyse an algorithm for minimising the simple regret in the
causal bandit problem.  

\begin{algorithm}[h]
\caption{Causal Best Arm Identification}\label{alg:simple}
\begin{algorithmic}[1]
\STATE {\bf Input:} $T, N$
\FOR{$t \in 1,\ldots,(T - 1) / 2$}
\STATE Choose the action $do()$ and observe $\boldsymbol{X}_t$ and $r_t$
\ENDFOR
\STATE Estimate $\mu$ using observation data:
\eq{
(\forall i,j) \qquad \hat \mu_{i,j} = \frac{\sum_{t=1}^{T/2} \ind{X_{t,i} = j} r_t}{\sum_{t=1}^{T/2} \ind{X_{t,i} = j}}\,.
}
\STATE Compute $\hat V_{ij} = \frac{T}{2 \sum_{t=1}^{T/2}\ind{X_{t,i}=j}} $
\STATE Compute $\hat m = m(\hat{\boldsymbol{V}})$
\STATE $i'(i) = $ the index of $\hat s_i$ in $\hat s'$
\STATE Compute $A$ as the subset of infrequently observed arms $\set{(i,j):V_{ij} \geq m}$ with $|A| = \hat m$

\FOR{$(i,j) \in A$}
\FOR{$t \in 1,\ldots,\frac{T - 1)}{2\hat m}$}
\STATE Choose action $do(X_{t,i} = j)$ and observe $r_t$
\ENDFOR
\STATE Recompute $\hat \mu_{i,j} = \frac{2\hat m}{T} \sum_{t=1}^{ T/2\hat m} r_t(X_{t,i}=j)$ 
\ENDFOR
\STATE Estimated optimal action is $\hat i^*, \hat j^* = \argmax_{i,j} \hat \mu_{i,j}$
\STATE Choose action $do(X_{t,\hat i^*} = \hat j^*)$
\end{algorithmic}
\end{algorithm}

\begin{theorem}\label{thm:uq-simple}
Algorithm \ref{alg:simple} satisfies
\eq{
\simpleregret \in \bigo{\sqrt{\frac{m}{T}\log\left(\frac{NT}{m}\right)}}\,.
}
\end{theorem}

Note that algorithms designed for finite-armed bandits would explore interventions more uniformly and achieve a regret of $\Omega(\sqrt{N/T})$.
Since $m$ is typically much smaller than $N$ we anticipate that the new algorithm will significantly outperform classical bandit algorithms in
this setting.
We prove a lower bound on the simple regret that matches up to logarithmic factors the upper bound given in Theorem \ref{thm:uq-simple}

\begin{theorem}\label{thm:lower}
Suppose $\boldsymbol{q}$ satisfies $m(\boldsymbol{q}) = m$.
Then for all strategies there exists a reward function such that
\eq{
\simpleregret \geq \frac{\frac{m}{2e} - 1}{m} \sqrt{\frac{2m}{3T}} = \Omega\left(\sqrt{\frac{m}{T}}\right)\,.
}
\end{theorem}

\ifsup
The proof of Theorems \ref{thm:uq-simple} and \ref{thm:lower} may be found in Sections \ref{sec:thm:uq-simple} and \ref{sec:thm:lower}.
\else
The proof of Theorems \ref{thm:uq-simple} and \ref{thm:lower} may be found in the supplementary material.
\fi



