

In this section we propose and analyse an algorithm for achieving the optimal regret in a natural special 
case of the causal bandit problem which we call the {\it parallel bandit}.
It is simple enough to admit a thorough analysis but rich enough to model the type of problem discussed in \S\ref{sec:intro}, including the farming example. 
It also suffices to witness the regret gap between algorithms that make use of causal models and those which do not.

The causal model for this class of problems has $N$ binary variables $\{ X_1, \ldots, X_N \}$ where each $X_i \in \{0,1\}$ are independent causes of a 
reward variable $Y \in \set{0,1}$, as shown in Figure~\ref{fig:causalStructure}.
All variables are observable and the set of allowable actions are all size 0 and size 1 interventions: 
\eq{
	\mathcal{A} = \set{do()} \cup \set{ do(X_i = j) \colon 1 \leq i \leq N \text{ and } j \in \set{0,1}}\,.
}
In the farming example from the introduction, $X_1$ might represent temperature (\eg, $X_1=0$ for low and $X_1=1$ for high). 
The interventions $do(X_1 = 0)$ and $do(X_1 = 1)$ indicate the use of shades or heat lamps to keep the temperature low or high, respectively.


\begin{figure}[h]
\centering
\begin{tikzpicture}[->,>=stealth',shorten >=1pt,auto,node distance=1cm,
  thick,main node/.style={observed}, hidden/.style={empty}]

 %nodes
\node[main node](1){$X_{1}$};
\node[main node, right=of 1](2){$X_{2}$};
\node[hidden, right=of 2](3){$...$};
\node[main node, right=of 3](4){$X_{N}$};
\node[main node, below right=of 2](5){Y};
 \path[every node/.style={font=\sffamily\small}]
    (1) edge (5)
    	(2) edge (5)
    (4) edge (5);
	
\end{tikzpicture}
\caption{Causal model for the parallel bandits problem.\label{fig:causalStructure}}
\end{figure}

In each round the learner either purely observes by selecting $do()$ or chooses a single variable to intervene on and the value to set it to. 
The remainder of the variables are simultaneously set by independently biased coin flips. 
The value of all variables are then used to determine the distribution of rewards for that round.
Formally, when not intervened upon we assume that each $X_i \sim \bernoulli(q_i)$ where $\vec{q} = (q_1, \ldots, q_N) \in [0,1]^N$ so that $q_i = \P{X_i = 1}$.
%\todom{Should $Y$ be 0-1 valued or real everywhere?} YES!
The value of the reward variable is distributed as $\P{Y = 1|\vec{X}} = r(\vec{X})$ where 
$r : \{0,1\}^N \to [0,1]$ is an arbitrary, fixed, and unknown function. 
In the farming example, this choice of $Y$ models the success or failure of a seasons crop, which depends stochastically on the various environment variables.
Note that the expected reward of the optimal intervention is at least as large as the expected reward for doing nothing.

%\todom{$i,j$ notation to $a \in \actions$}
%In our analysis below, we will often use the pair $(i,j)$ for $i \in \{1, \ldots, N\}$ and $j \in \{0,1\}$ as a shorthand for the intervention $do(X_i = j)$.
%We will also make use of the expansion of the expected reward for $do(X_i = j)$:
%\eq{
%\mu_{i,j} 
%&= \E{r(X)|do(X_i = j)} \\
%&= \sum_{\boldsymbol{x} \in \set{0,1}^N : x_i = j} r(\boldsymbol{x})  \prod_{k \neq i} q_k^{x_k} (1 - q_k)^{1-x_k}\,.  
%}
%The sum and product use the restrictions $x_i = j$ and $k \ne i$ because the intervention forces $X_i = j$ with probability 1.
%The optimal intervention is denoted $(i^*,j^*) = \argmax_{i,j} \mu_{i,j}$ and the corresponding optimal reward is $\mu^* = \mu_{i^*,j^*}$. 



\subsection{The Parallel Bandit Algorithm}
\label{sub:par-bandit-alg}

The algorithm operates as follows. For the first $T/2$ rounds it chooses $do()$ to collect observational data.
The fact that $X_1,\ldots,X_N$ are independent can be used to create good estimators of the returns of those
actions $do(X_i = j)$ for which $\P{X_i = j}$ is large. These are the interventions that are essentially observed
from observational data. Those actions where $\P{X_i = j}$ is small may not be observed (often) in the first $T/2$ rounds, so
the estimates of the returns of these actions from the first $T/2$ samples could be quite poor. For the remaining $T/2$ rounds
the algorithm makes interventions on the actions $a = do(X_i = j)$ for which $\P{X_i = j}$ is small. The precise allocation is
carefully chosen to minimise the worst-case simple regret. The difficulty of the problem depends on $\vec{q}$. 
Define $m \in \set{1, \ldots, N}$ by
\eq{
m(\vec{q}) = \max \set{i : \sum_{k=1}^N \ind{\min\set{q_k, (1 - q_k)}} \leq 1/i}\,.
}

\begin{algorithm}[h]
\caption{Parallel Bandit Algorithm}\label{alg:simple}
\begin{algorithmic}[1]
\STATE {\bf Input:} Total rounds $T$ and $N$.
\FOR{$t \in 1,\ldots,T / 2$}
\STATE Perform empty intervention $do()$
\STATE Observe $\vec{X}_t$ and $Y_t$
\ENDFOR
\FOR{$a = do(X_i = x) \in \actions$}
\STATE Count times $X_i = x$ seen: $T_a = \sum_{t=1}^{T/2} \ind{X_{t,i} = x}$
\STATE Estimate reward: $\hat{\mu}_a = \frac{1}{T_a} \sum_{t=1}^{T/2} \ind{X_{t,i} = x} Y_t$ \\[0.2cm]
\STATE Estimate probabilities: $\hat{p}_a = \frac{2 T_a}{T}$ 
\STATE $\hat q_i = \hat{q}_{do(X_i = 1)}$
\ENDFOR
\STATE Compute $\hat{m} = m(\vec{\hat q})$ and $A = \set{a \in \actions \colon \hat{p}_a \leq 1/\hat m}$.
\STATE Let $T_A := \frac{T}{2 |A|}$ be times to sample each $a\in A$.
\FOR{$a = do(X_i = x) \in A$}
\FOR{$t \in 1,\ldots,T_A$}
\STATE Intervene with $a$ and observe $Y_t$
\ENDFOR
\STATE Re-estimate $\hat{\mu}_a = \frac{1}{T_A} \sum_{t=1}^{T_A} Y_t$
\ENDFOR
\RETURN estimated optimal $\hat{a}^* \in \argmax_{a\in\actions} \hat{\mu}_a$
\end{algorithmic}
\end{algorithm}

\begin{theorem}\label{thm:uq-simple}
Algorithm \ref{alg:simple} satisfies
\eq{
\simpleregret \in \bigo{\sqrt{\frac{m(\vec{q})}{T}\log\left(\frac{NT}{m}\right)}}\,.
}
\end{theorem}

Note that algorithms designed for finite-armed bandits would explore interventions more uniformly and achieve a regret of $\Omega(\sqrt{N/T})$.
Since $m$ is typically much smaller than $N$ we anticipate that the new algorithm will significantly outperform classical bandit algorithms in
this setting.
We prove a lower bound on the simple regret that matches up to logarithmic factors the upper bound given in Theorem \ref{thm:uq-simple}

\begin{theorem}\label{thm:lower}
Suppose $\boldsymbol{q}$ satisfies $m(\boldsymbol{q}) = m$.
Then for all strategies there exists a reward function such that
\eq{
\simpleregret \geq \frac{\frac{m}{2e} - 1}{m} \sqrt{\frac{2m}{3T}} = \Omega\left(\sqrt{\frac{m}{T}}\right)\,.
}
\end{theorem}

\ifsup
The proof of Theorems \ref{thm:uq-simple} and \ref{thm:lower} may be found in Sections \ref{sec:thm:uq-simple} and \ref{sec:thm:lower}.
\else
The proof of Theorems \ref{thm:uq-simple} and \ref{thm:lower} may be found in the supplementary material.
\fi



