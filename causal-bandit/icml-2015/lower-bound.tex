
We prove a lower bound on the simple regret that matches up to logarithmic factors the upper bound given in \cref{thm:simple}

\todot{Remark that the Gaussian assumption is convenient but that other noise models will also work. Eg., Bernoulli.}
\begin{theorem}\label{thm:lower}
Suppose $\eta_t \sim \mathcal N(0, 1)$ and $\boldsymbol{q}$ satisfies $m(\boldsymbol{q}) = m$.
Then for all strategies there exists a reward function such that
\eq{
\simpleregret \geq \frac{\frac{m}{2e} - 1}{m} \sqrt{\frac{2m}{3T}} = \Omega\left(\sqrt{\frac{m}{T}}\right)\,.
}
\end{theorem}

\begin{proof}
Assume without loss of generality that $q_i \leq 1/2$ for all $i$ and that $q_1 \leq q_2 \leq \ldots \leq q_N$.
For each $i$ define reward function $r_i$ by
\eq{
r_0(\boldsymbol{X}) &= 0 &
r_i(\boldsymbol{X}) &= \begin{cases}
\epsilon & \text{if } X_i = 1 \\
0 & \text{otherwise}\,,
\end{cases}
}
where $\epsilon > 0$ is some constant to be chosen later.
We abbreviate $R_{T,i}^{\text{simple}}$ to be the expected simple regret incurred when interacting with the
environment determined by $\boldsymbol{q}$ and $r_i$. Let $P_i$ be the corresponding measure
on all observations over all $T$ rounds and $\mathbb E_i$ the expectation with respect to $P_i$. By Lemma 2.6 by \cite{Tsy08} we have
\eq{
\Prz{I_t = i} + \Pri{I_t \neq i} \geq \exp\left(-\KL(P_0, P_i)\right)\,,
}
where $\KL(P_0, P_i)$ is the KL divergence between measures $P_0$ and $P_i$.
Let $T_i(T) = \sum_{t=1}^T \ind{I_t = i}$ be the total number of times the learner intervenes on variable $i$ (with any intervention). 
Then for $i \leq m$ a computation shows that 
\todot{Need to add at minimum some citations here}
\eq{
\KL(P_0, P_i) 
&= \frac{\epsilon^2}{2} \mathbb E_0\left[\sum_{t=1}^T \ind{X_{t,i} = 1}\right] \\
&\leq \frac{\epsilon^2}{2} \left(\mathbb E_0 T_i(T) + q_i T\right) \\
&\leq \frac{\epsilon^2}{2} \left(\mathbb E_0 T_i(T) + \frac{T}{m}\right)\,.
}
Define set $A$ by
\eq{
A = \set{i \leq m : \mathbb E_0 T_i(T) \leq 2T / m}\,.
}
Then for $i \in A$ and choosing $\epsilon = \sqrt{2m/3T}$ we have
\eq{
\KL(P_0, P_i) \leq \frac{3T\epsilon^2}{2m} = 1\,. 
}
Now $\sum_{i=1}^m \mathbb E_0 T_i(T) \leq T$, which implies that $|A| \geq m/2$.
Therefore
\eq{
\sum_{i \in A} \Pri{I_t \neq i} 
&\geq \sum_{i \in A} \exp\left(-\KL(P_0, P_i)\right) - 1\\
&\geq \frac{|A|}{e} - 1 
\geq \frac{m}{2e} - 1\,.
}
Therefore there exists an $i \in A$ such that
\eq{
\Pri{I_t \neq i} \geq \frac{\frac{m}{2e} - 1}{m}\,.
}
Therefore
\eq{
R^{\text{simple}}_{T,i} \geq \frac{1}{2} P_i(I_t \neq i) \epsilon \geq \frac{\frac{m}{2e} - 1}{2m} \sqrt{\frac{2m}{3T}}
}
as required.
\end{proof}

