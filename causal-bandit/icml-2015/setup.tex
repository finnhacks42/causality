\newcommand{\bernoulli}{\operatorname{Bernoulli}}
\newcommand{\dirac}{\operatorname{Dirac}}
\renewcommand{\vec}[1]{\boldsymbol{#1}}

We now introduce a novel class of stochastic sequential decision problems which we call \emph{causal bandit problems}. In these problems, rewards are given for repeated interventions on a fixed causal model~\cite{Pearl2000}. 
The \emph{causal model} is specified using a directed acyclic graph over a set of random variables $\mathcal{X}$ that defines a Bayesian network over $\mathcal{X}$ where an edge from variable $X$ to $X'$ is interpreted to mean that a change in the value of $X$ directly causes a change to the value of $X'$. 
An \emph{intervention (of size $n$)}, denoted $do(\vec{X}=\vec{x})$, assigns the values $\vec{x}=\{x_1, \ldots, x_n\}$ to the corresponding variables $\vec{X}=\{X_1, \ldots, X_n\}$ with the empty intervention (where no variable is set) denoted $do()$.
An intervention ``mutilates'' the original graph by removing edges into the variables in $\vec{X}$ and the resulting graph defines a probability distribution $\P{\vec{X}^c | do(\vec{X}=\vec{x})}$ over $\vec{X}^c := \mathcal{X} - \vec{X}$. 
Further details can be found in Chapter 21 of~\cite{Koller2009}.

A learner for a casual bandit problem is given the casual model, a set of \emph{observable} variables $\mathcal{O} \subseteq \mathcal{X}$, a set of \emph{settable} variables $\mathcal{S} \subset \mathcal{X}$, a \emph{reward variable} $Y \in \mathcal{O} \cap \mathcal{S}^c$, and the following game is played over $T$ rounds. In round $t$, the learner selects an intervention $do(\vec{X}_t = \vec{x}_t)$ for $\vec{X}_t \subseteq \mathcal{S}$. Values for all non-intervened variables $\vec{X}^c_t$ are sampled from $\P{\vec{X}^c_t | do(\vec{X}_t = \vec{x}_t)}$ and the learner is shown the sampled values of all the observable variables in $\vec{X}^c_t \cap \mathcal{O}$, and recevies the reward $Y_t$.
We note that classical $K$-armed stochastic bandit problem can be recovered in our framework by considering a causal model with an edge connecting a single settable variable $X$ that can take on $K$ values to a reward variable $Y = r(X) + \eta$ for some arbitrary but unknown, real-valued function $r$ and a noise term $\eta$ that may depend on $X$.

The expected reward for playing $do(\vec{X}_t = \vec{x}_t)$ is $\mu_{\vec{x}_t} := \E{Y_t | do(\vec{X}_t = \vec{x}_t)}$.

The \emph{cumulative regret} measures the difference between the reward expected under
the omnipotent strategy that knows the optimal action in advance and the expected reward of the learner and
is denoted
\eqn{
\label{eq:regret}
R_T = T \mu^* - \E{\sum_{t=1}^T Y_t}\,.
}
The second performance measure is the \emph{simple regret}, which measures the performance of the learner on the final round only.
\eqn{
\label{eq:regret-simple}
\simpleregret = \mu^* - \E{\mu_{I_T,J_T}}\,.
}
The simple regret typically makes more sense when the learning agent has a fixed learning budget after which its policy will be fixed indefintely. The cumulative
regret is more appropriate for online agents that may continue to adapt their policy.


\subsection{Simple Causal Bandit Problem}

In the remainder of this paper we focus on a causal model that is sufficient to demonstrate the separation of regret bounds between algorithms that make use of the additional causal information and those which do not.

Assume we have a known causal model with binary variables $\boldsymbol{X} = \{X_{1},\ldots,X_{N}\}$ that independently cause a 
target variable of interest $Y \in \R$ (see Figure \ref{fig:causalStructure}).
\begin{figure}[h]
\centering
\caption{Assumed Causal Structure}
\label{fig:causalStructure}
\begin{tikzpicture}[->,>=stealth',shorten >=1pt,auto,node distance=1cm,
  thick,main node/.style={observed}, hidden/.style={empty}]

 %nodes
\node[main node](1){$X_{1}$};
\node[main node, right=of 1](2){$X_{2}$};
\node[hidden, right=of 2](3){$...$};
\node[main node, right=of 3](4){$X_{N}$};
\node[main node, below right=of 2](5){Y};
 \path[every node/.style={font=\sffamily\small}]
    (1) edge (5)
    	(2) edge (5)
    (4) edge (5);
	
\end{tikzpicture}
\end{figure}


The game proceeds over $T$ identical rounds (or time-steps).
In each round $t$ the learner can choose either to do nothing (denoted by $I_t = \bot$) or they can choose $I_t \in \set{1,\ldots,N}$ and
an intervention $J_t \in \set{0,1}$. After the learner has chosen an intervention they observe $X_{t,i} \in \set{0,1}$ for all $i$ where
\eq{
X_{t,i} \sim \begin{cases}
\dirac(J_t) & \text{if } I_t = i \\
\bernoulli(q_i) & \text{otherwise}\,.
\end{cases}
}
where $\boldsymbol{q} \in [0,1]^N$ is a (possibly unknown) vector of probabilities with $q_i = \P{X_i = 1}$. 
Note that if the learner did not choose an intervention ($I_t =  \bot$), then
$X_{t,i} \sim \bernoulli(q_i)$ for all variables $i \in \set{1,\ldots,N}$.
Finally the learner observes the reward $Y_t = r(X_t) + \eta_t$ where 
\eq{
r : \set{0,1}^N \to \R
}
is arbitrary (and unknown) 
and $\eta_t$ is a $1$-subgaussian noise
term (with the distribution possibly dependent on $X_t$). 
In the farming example $X_{t,1}$ might represent whether the plants had sufficient water in the $t$th season (controllable by installing irrigation) 
while $X_{t,2}$ might be whether or not there was a spring frost (controllable with heat-lamps or covers).

The expected reward for intervening on variable $i$ by setting it to $j$ is defined by
\eq{
\mu_{i,j} 
&= \E{r(X)|do(X_i = j)} \\
&= \sum_{\boldsymbol{x} \in \set{0,1}^N : x_i = j} r(\boldsymbol{x})  \prod_{k \neq i} q_k^{x_k} (1 - q_k)^{1-x_k} \,. 
}
The optimal intervention is $(i^*,j^*) = \argmax_{i,j} \mu_{i,j}$ and the corresponding optimal reward is $\mu^* = \mu_{i^*,j^*}$. 
Note that the expected reward of the optimal intervention is at least as large as the expected reward for doing nothing.

It is worth mentioning that the problem may be treated as a multi-armed bandit with $2N$ arms, one corresponding to each intervention.
As we shall shortly see, this approach is usually not practical because the resulting algorithms do not exploit the structure in
the covariates.


\begin{remark}
In order to be consistent with the literature on causal inference we use the notation $do()$ to denote the action of doing nothing and $do(X_{t,I_t} = J_t)$
the action of intervening on the $I_t$th variable and setting it to equal $J_t$. 
In the bandit community it is implicit that 
algorithms selecting actions are intervening in the system. So it is sufficient to index actions according to the variable and value. 
However, in causal graphs, it is essential to differentiate observing (or conditioning) on a variable taking a certain value, from 
intervening to set that variable. Although in the specific causal graph we consider, observation and intervention are the same, we 
deliberately introduce the do-notation \cite{Pearl2000} that makes this distinction clear so as to help provide a bridge between the 
bandit and causal inference communities.
\end{remark}


