

%%%%%%%%%%%%%%%%%%%%%%%%%%%%%%%%%%%%%%%%%%%%
% PROOF OF SIMPLE REGRET UPPER BOUND
%%%%%%%%%%%%%%%%%%%%%%%%%%%%%%%%%%%%%%%%%%%%
\section{Proof of Theorem \ref{thm:uq-simple}}\label{sec:thm:uq-simple}

Assume without loss of generality that $q_1 \leq q_2 \leq \ldots \leq q_N \leq 1/2$. The assumption is non-restrictive since all variables
are independent and permutations of the variables can be pushed to the reward function.
The proof of Theorem \ref{thm:uq-simple} requires some lemmas, the first of which is an immediate consequence of the Chernoff bound.


\begin{lemma}\label{lem:conc1}
Let $i \in \set{1,\ldots, N}$ and $\delta > 0$. Then
\eq{
\P{\left|\hat q_i - q_i\right| \geq \sqrt{\frac{6q_i}{T} \log \frac{2}{\delta}}} \leq \delta\,.
}
\end{lemma}

\begin{lemma}\label{lem:conc2}
Let $X_1,X_2\ldots,$ be a sequence of random variables with $X_i \in [0,1]$ and $\EE X_i = p$ and $\delta \in [0,1]$.
Then 
\eq{
\P{\exists t \geq n_0 : \left|\frac{1}{t} \sum_{s=1}^t X_s - p\right| \geq \sqrt{\frac{2}{n_0} \log \frac{2}{\delta}}} \leq 4\delta\,.
}
\end{lemma}

\begin{proof}
For $\delta \geq 1/4$ the result is trivial. Otherwise 
by Hoeffding's bound and the union bound:
\eq{
\P{\exists t \geq n_0 : \left|\frac{1}{t} \sum_{s=1}^t X_s - p\right| \geq \sqrt{\frac{2}{n_0} \log \frac{2}{\delta}}} 
&\leq \sum_{t = n_0}^\infty \P{\left|\frac{1}{t} \sum_{s=1}^t X_s - p\right| \geq \sqrt{\frac{2}{n_0} \log \frac{2}{\delta}}} \\
&\leq 2\sum_{t=n_0}^\infty \exp\left(-\frac{t}{n_0} \log \frac{2}{\delta}\right) 
\leq 4\delta\,. \qedhere
}
\end{proof}



\begin{lemma}\label{lem:m_est}
Let $\delta \in (0,1)$ and assume $T \geq 48m \log\frac{2N}{\delta}$. Then
\eq{
\P{2m(\vec{q}) / 3 \leq m(\vec{\hat q}) \leq 2m(\vec{q})} \geq 1 - \delta\,.
}
\end{lemma}

\begin{proof}
Let $F$ be the event that there exists and $1 \leq i \leq N$ for which
\eq{
\left|\hat q_i - q_i\right| \geq \sqrt{\frac{6q_i}{T} \log \frac{2N}{\delta}}\,.
}
Then by the union bound and Lemma \ref{lem:conc1} we have $\P{F} \leq \delta$. The result will be completed by showing that
when $F$ does not hold we have $2m(\vec{q})/3 \leq m(\vec{\hat q}) \leq 2m(\vec{q})$.
Towards this end recall that $m(\vec{q}) = \min\set{i : q_i \geq 1 / i}$.
Therefore for $i > m$ we have $q_i \geq q_m \geq 1/m$ and so by Lemma \ref{lem:conc1} we have
\eq{
\frac{3}{4} 
&\geq \frac{1}{2} + \sqrt{\frac{3}{T} \log \frac{2N}{\delta}} 
\geq q_i + \sqrt{\frac{6q_i}{T} \log \frac{2N}{\delta}} 
\geq \hat q_i \\
&\geq q_i - \sqrt{\frac{6q_i}{T} \log \frac{2N}{\delta}}
\geq q_i - \sqrt{\frac{q_i}{8m}}
\geq \frac{1}{2m}\,.
}
Therefore by the pigeonhole principle we have $m(\vec{\hat q}) \leq 2m$.
For the other direction we proceed in a similar fashion. Since the failure event $F$ does not hold we have for $i \leq m$ that
\eq{
\hat q_i 
\leq q_i + \sqrt{\frac{6q_i}{T} \log\frac{2N}{\delta}} 
\leq \frac{1}{m} \left(1 + \sqrt{\frac{1}{8}}\right)
\leq \frac{3}{2m}\,.
}
Therefore $m(\vec{\hat q}) \geq 2m(\vec{q}) / 3$ as required. 
\end{proof}

\begin{proof}[Proof of Theorem \ref{thm:uq-simple}]
Let $\delta = m = m(\vec{q}) / N$. Then by Lemma \ref{lem:m_est} we have 
\eq{
\P{2m/3 \leq m(\vec{\hat q}) \leq 2m} \geq 1 - \delta\,.
}
Recall that $A = \set{a \in \actions : \hat p_a \leq 1/m(\vec{\hat q})}$. Then
for $a \in A$ the algorithm estimates $\mu_a$ from $T/(2m(\vec{\hat q})) \geq T/(4m)$ samples.
Therefore by Hoeffding's inequality and the union bound we have
\eq{
\P{\exists a \in A : |\mu_a - \hat \mu_a| \geq \sqrt{\frac{8m}{T} \log\frac{2N}{\delta}}} \leq \delta\,.
}
For arms not in $a$ we have $\hat p_a \geq 1/m(\vec{\hat q}) \geq 1/(2m)$.
Therefore if $a = do(X_i = j)$, then 
\eq{
\hat p_a = \frac{2}{T} \sum_{t=1}^{T/2} \ind{X_i = j} \geq \frac{1}{2m}\,. 
}
Therefore $\sum_{t=1}^{T/2} \ind{X_{t,i} = j} \geq T/4m$
and by Lemma \ref{lem:conc2} we have
\eq{
\P{\sum_{t=1}^{T/2} \ind{X_i = j} \geq \frac{T}{4m} \text{ and } \left|\hat \mu_a - \mu_a\right| \geq \sqrt{\frac{8m}{T} \log \frac{2N}{\delta}}} \leq 4\delta / N\,.
}
Therefore with probability at least $1 - 6\delta$ we have
\eq{
(\forall a \in \actions) \qquad |\hat \mu_a - \mu_a| \leq \sqrt{\frac{8m}{T} \log \frac{N}{\delta}} = \epsilon\,.
}
If this occurs, then 
\eq{
\mu_{\hat a^*} \geq \hat \mu_{\hat a^*} - \epsilon \geq \hat \mu_{a^*} - \epsilon \geq \mu_{a^*} - 2\epsilon\,.
}
Therefore
\eq{
\mu^* - \EE[\mu_{\hat a^*}] 
\leq 6\delta + \epsilon 
\leq \frac{6m}{T} + \sqrt{\frac{32m}{T} \log \frac{NT}{m}}\,, 
}
which completes the result.
\end{proof}

%%%%%%%%%%%%%%%%%%%%%%%%%%%%%%%%%%%%%%%%%%%%
% LOWER BOUND
%%%%%%%%%%%%%%%%%%%%%%%%%%%%%%%%%%%%%%%%%%%%
\section{Proof of Theorem \ref{thm:lower}}\label{sec:thm:lower}

\todot{Change this to Bernoulli}
Assume without loss of generality that $q_1 \leq q_2 \leq \ldots \leq q_N \leq 1/2$.
For each $i$ define reward function $r_i$ by
\eq{
r_0(\boldsymbol{X}) &= \frac{1}{2} &
r_i(\boldsymbol{X}) &= \begin{cases}
\frac{1}{2} + \epsilon & \text{if } X_i = 1 \\
\frac{1}{2} & \text{otherwise}\,,
\end{cases}
}
where $\epsilon > 0$ is some constant to be chosen later.
We abbreviate $R_{T,i}$ to be the expected simple regret incurred when interacting with the
environment determined by $\boldsymbol{q}$ and $r_i$. Let $P_i$ be the corresponding measure
on all observations over all $T$ rounds and $\mathbb E_i$ the expectation with respect to $P_i$. By Lemma 2.6 by \cite{Tsy08} we have
\eq{
\Prz{I_T = i} + \Pri{I_T \neq i} \geq \exp\left(-\KL(P_0, P_i)\right)\,,
}
where $\KL(P_0, P_i)$ is the KL divergence between measures $P_0$ and $P_i$.
Let $T_i(T) = \sum_{t=1}^T \ind{I_t = i}$ be the total number of times the learner intervenes on variable $i$ (with any intervention). 
Then for $i \leq m$ a computation shows that 
\todot{Need to add at minimum some citations here}
\eq{
\KL(P_0, P_i) 
&\leq 4\epsilon^2 \EE_0\left[\sum_{t=1}^T \ind{X_{t,i} = 1}\right] \\
%&= \frac{\epsilon^2}{2} \mathbb E_0\left[\sum_{t=1}^T \ind{X_{t,i} = 1}\right] \\
&\leq 4\epsilon^2 \left(\mathbb E_0 T_i(T) + q_i T\right) \\
&\leq 4\epsilon^2 \left(\mathbb E_0 T_i(T) + \frac{T}{m}\right)\,.
}
Define set $A$ by
\eq{
A = \set{i \leq m : \mathbb E_0 T_i(T) \leq 2T / m}\,.
}
Then for $i \in A$ and choosing $\epsilon = \sqrt{2m/3T}$ we have
\eq{
\KL(P_0, P_i) \leq \frac{3T\epsilon^2}{2m} = 1\,. 
}
Now $\sum_{i=1}^m \mathbb E_0 T_i(T) \leq T$, which implies that $|A| \geq m/2$.
Therefore
\eq{
\sum_{i \in A} \Pri{I_T \neq i} 
&\geq \sum_{i \in A} \exp\left(-\KL(P_0, P_i)\right) - 1\\
&\geq \frac{|A|}{e} - 1 
\geq \frac{m}{2e} - 1\,.
}
Therefore there exists an $i \in A$ such that
\eq{
\Pri{I_T \neq i} \geq \frac{\frac{m}{2e} - 1}{m}\,.
}
Therefore
\eq{
R_{T,i} \geq \frac{1}{2} P_i(I_T \neq i) \epsilon \geq \frac{\frac{m}{2e} - 1}{2m} \sqrt{\frac{2m}{3T}}
}
as required.










