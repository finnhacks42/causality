%%%%%%%%%%%%%%%%%%%%%%%%%%%%%%%%%%%%%%%%%

%
%%%%%%%%%%%%%%%%%%%%%%%%%%%%%%%%%%%%%%%%%

%----------------------------------------------------------------------------------------
%	PACKAGES AND OTHER DOCUMENT CONFIGURATIONS
%----------------------------------------------------------------------------------------
\documentclass[12pt]{beamer}

\usepackage[scale=1.3]{beamerposter} % Use the beamerposter package for laying out the poster
\usepackage{graphicx} % more modern 
\usepackage{caption}
\usepackage{subcaption}
\usepackage{tikz}
\usepackage{dsfont}
\usepackage{amsmath}
\usepackage{amssymb}
\usepackage{amsthm}
\usepackage{times}
\usepackage{natbib}
\usepackage{xcolor}
\usepackage[noend]{algorithmic}
\algsetup{indent=6em}
\usepackage{algorithm}

\usetheme{confposter} % Use the confposter theme supplied with this template

\setbeamerfont{itemize/enumerate subbody}{size=\normalsize} %to set the body size
\setbeamertemplate{itemize subitem}{\normalsize\raise1.25pt\hbox{\donotcoloroutermaths$\blacktriangleright$}}  %to set the symbol size

\setbeamertemplate{itemize item}{\normalsize\raise1.25pt\hbox{\donotcoloroutermaths$\blacktriangleright$}}  %to set the symbol size

\setbeamercolor{block title}{fg=ngreen,bg=white} % Colors of the block titles
\setbeamercolor{block body}{fg=black,bg=white} % Colors of the body of blocks
\setbeamercolor{block alerted title}{fg=white,bg=dblue!70} % Colors of the highlighted block titles
\setbeamercolor{block alerted body}{fg=black,bg=dblue!10} % Colors of the body of highlighted blocks
% Many more colors are available for use in beamerthemeconfposter.sty

%-----------------------------------------------------------
% Define the column widths and overall poster size
% To set effective sepwid, onecolwid and twocolwid values, first choose how many columns you want and how much separation you want between columns
% In this template, the separation width chosen is 0.024 of the paper width and a 4-column layout
% onecolwid should therefore be (1-(# of columns+1)*sepwid)/# of columns e.g. (1-(4+1)*0.024)/4 = 0.22
% Set twocolwid to be (2*onecolwid)+sepwid = 0.464
% Set threecolwid to be (3*onecolwid)+2*sepwid = 0.708

\newlength{\sepwid}
\newlength{\onecolwid}
\newlength{\twocolwid}
\newlength{\threecolwid}
\setlength{\paperwidth}{48in} % A0 width: 46.8in
\setlength{\paperheight}{36in} % A0 height: 33.1in
\setlength{\sepwid}{0.024\paperwidth} % Separation width (white space) between columns
\setlength{\onecolwid}{0.3\paperwidth} % Width of one column
\setlength{\twocolwid}{0.464\paperwidth} % Width of two columns
\setlength{\threecolwid}{0.708\paperwidth} % Width of three columns
\setlength{\topmargin}{-0.5in} % Reduce the top margin size
%-----------------------------------------------------------

\usepackage{graphicx}  % Required for including images

\usepackage{booktabs} % Top and bottom rules for tables



\newcommand{\defined}{\vcentcolon =}
\newcommand{\rdefined}{=\vcentcolon}
\newcommand{\E}[1]{\mathbb E\left[#1\right]}
\newcommand{\R}{\mathbb R}
\newcommand{\Var}{\operatorname{Var}}
\newcommand{\calF}{\mathcal F}
\newcommand{\sr}[1]{\stackrel{#1}}
\newcommand{\set}[1]{\left\{#1\right\}}
\newcommand{\ind}[1]{\mathds{1}\!\!\set{#1}}
%\newcommand{\ind}[1]{\set{#1}}
\newcommand{\argmax}{\operatornamewithlimits{arg\,max}}
\newcommand{\argmin}{\operatornamewithlimits{arg\,min}}
\newcommand{\floor}[1]{\left \lfloor {#1} \right\rfloor}
\newcommand{\ceil}[1]{\left \lceil {#1} \right\rceil}
\newcommand{\eqn}[1]{\begin{align}#1\end{align}}
\newcommand{\eq}[1]{\begin{align*}#1\end{align*}}
\newcommand{\Ber}{\operatorname{Bernoulli}}
\renewcommand{\P}[1]{\operatorname{P}\left\{#1\right\}}
\newcommand{\Pri}[1]{\operatorname{P}_i\left\{#1\right\}}
\newcommand{\Prz}[1]{\operatorname{P}_0\left\{#1\right\}}
\newcommand{\bigo}[1]{\mathcal{O}\left( #1 \right)}
\newcommand{\bigotilde}[1]{\tilde{\mathcal{O}}\left( #1 \right)}
\newcommand{\bigtheta}[1]{\Theta\left( #1 \right)}
\newcommand{\bigthetatilde}[1]{\tilde{\Theta}\left( #1 \right)}
\newcommand{\bigomega}[1]{\Omega\left( #1 \right)}
\newcommand{\KL}{\operatorname{KL}}
\newcommand{\simpleregret}{R_T}
\newcommand{\Pij}[1]{\operatorname{P}_{ij}\!\left\{#1\right\}}
\newcommand{\Pkl}[1]{\operatorname{P}_{kl}\!\left\{#1\right\}}
\newcommand{\Q}[1]{\operatorname{Q}\left\{#1\right\}}
\newcommand{\EE}{\mathbb E}
\newcommand{\EEa}{\EE_a}
\newcommand{\Pns}[2]{\operatorname{P}_{#1}\left\{#2\right\}}
\newcommand{\Pn}[2]{\operatorname{P}\left\{#2|#1\right\}}
\newcommand{\parents}[1]{\operatorname{\mathcal{P}a}_{#1}}
\newcommand{\actions}{\mathcal{A}}
\newcommand{\calA}{\mathcal A}
\newcommand{\etc}{\textit{etc}}
\newcommand{\ie}{\textit{i.e.}}
\newcommand{\eg}{\textit{e.g.}}
\newcommand{\calP}{\mathcal P}
\newcommand{\x}{\boldsymbol{x}}
\newcommand{\Ps}{\operatorname{P}}


\let\temp\epsilon
\let\epsilon\varepsilon
\let\vec\mathbf

\graphicspath{ {figures/} }
\usetikzlibrary{arrows,positioning,backgrounds} 
\tikzset{
    %Define standard arrow tip
    >=stealth',
    %Define style for boxes
    observed/.style={
           circle,
           rounded corners,
           draw=black, thick,
           minimum width=2.0em,
           minimum height=2.0em,
           font=\tiny,
           text centered,
           scale=.8,
           fill=blue!20!white},
     latent/.style={
           circle,
           rounded corners,
           draw=black, thick, dashed,
           minimum width=.5em,
           minimum height=.5em,
           font=\footnotesize,
           text centered,
           fill=black!10!white
           },
     empty/.style={
           circle,
           rounded corners,
           minimum width=.5em,
           minimum height=.5em,
           font=\footnotesize,
           text centered,
           },
    % Define arrow style
    pil/.style={
           o->,
           thick,
           shorten <=2pt,
           shorten >=2pt,},
    sh/.style={ shade, shading=axis, left color=red, right color=green,
    shading angle=45 }  
}

%----------------------------------------------------------------------------------------
%	TITLE SECTION 
%----------------------------------------------------------------------------------------

\title{Causal Bandits: Learning Good Interventions via Causal Inference} % Poster title

\author{Finnian Lattimore*, Tor Lattimore** and Mark Reid*} % Author(s)

\institute{*Australian National University and Data61, ** Indiana University Bloomington} % Institution(s)

%----------------------------------------------------------------------------------------

\begin{document}

\addtobeamertemplate{block end}{}{\vspace*{2ex}} % White space under blocks
\addtobeamertemplate{block alerted end}{}{\vspace*{2ex}} % White space under highlighted (alert) blocks

\setlength{\belowcaptionskip}{12pt} % White space under figures
%\setlength\belowdisplayshortskip{2ex} % White space under equations

%\setlength\abovedisplayskip{40pt}
%\setlength\belowdisplayskip{40pt}
%\setlength\abovedisplayshortskip{40pt}
%\setlength\belowdisplayshortskip{40pt}

\begin{frame}[t] % The whole poster is enclosed in one beamer frame

\begin{columns}[t] % The whole poster consists of three major columns, the second of which is split into two columns twice - the [t] option aligns each column's content to the top

\begin{column}{\sepwid}\end{column} % Empty spacer column

\begin{column}{.8\onecolwid} % The first column


\begin{block}{Introduction}

We study the problem of using causal models to improve the rate at which good interventions can be learned online in a stochastic environment. 
Our formalism combines multi-arm bandits and causal inference to model a novel type of bandit feedback that is not exploited by existing approaches.
\end{block}

%------------------------------------------------
\begin{block}{Setup}
\begin{itemize}
\setlength\itemsep{28pt}
\item Learner given causal graph $\mathcal{G}$ over a set of random variables $\mathcal{X}$
\item Actions assign values to subsets of variables in $\mathcal{G}$. The set of allowed actions/arms $\mathcal{A}$ is a subset of all possible assignments.
\item One variable $Y \in \mathcal{X}$ is designated as the \emph{reward variable} and takes on values in $\{0, 1\}$.
\item The game proceeds over $T$ rounds. In round $t$, the learner takes action $a_t = do(\vec{X}_t = \vec{x}_t) \in \mathcal{A}$, \textbf{\emph{then}} observes reward $Y_t$ and the value of all other variables in $\mathcal{X}$.
\item Simple regret: goal is to minimize $\simpleregret = \mu^* - \E{\mu_{\hat a^*_T}}$, where $\mu^*$ is the expected reward of the best action and $\hat a^*_T$ is the action the learner estimates to be optimal at time $T$
\end{itemize}


\end{block}

\begin{block}{Parallel Model}

\begin{figure}
    \begin{subfigure}[b]{0.45\textwidth}
	\centering    
          \begin{tikzpicture}[->,>=stealth',shorten >=1pt,auto,node distance=1.0em,
  thick,main node/.style={observed}, hidden/.style={empty},background rectangle/.style={fill=olive!45}]
%every node/.style={scale=0.6}
 %nodes
\node[main node](1){$X_{1}$};
\node[main node, right=of 1](2){$X_{2}$};
\node[hidden, right=of 2](3){$...$};
\node[main node, right=of 3](4){$X_{N}$};
\node[main node, below right=of 2](5){$Y$};
 \path[every node/.style={font=\tiny}]
    (1) edge (5)
    	(2) edge (5)
    (4) edge (5);
\end{tikzpicture}
        \caption{Parallel graph}
        \label{fig:parallel}
    \end{subfigure}
   \begin{subfigure}[b]{0.45\textwidth}
	\centering    
          \begin{tikzpicture}[->,>=stealth',shorten >=1pt,auto,node distance=1.0em,
  thick,main node/.style={observed}, hidden/.style={empty},background rectangle/.style={fill=olive!45}]
%every node/.style={scale=0.6}
 %nodes
\node[main node](1){$x$};
\node[main node, right=of 1](2){$X_{2}$};
\node[hidden, right=of 2](3){$...$};
\node[main node, right=of 3](4){$X_{N}$};
\node[main node, below right=of 2](5){$Y$};
 \path[every node/.style={font=\tiny}]
    (1) edge (5)
    	(2) edge (5)
    (4) edge (5);
\end{tikzpicture}
        \caption{after $do(X_1=x)$}
        \label{fig:parallel_post_action}
    \end{subfigure}
\caption{Intervention in the Parallel Model}
\label{fig:parallel_model} 
\end{figure}
%\vspace{-1cm}
\begin{itemize}
\setlength\itemsep{28pt}
\item Causal graph as in figure \ref{fig:parallel} with all variables binary.
\item $\mathcal{A} = \set{do()} \cup \set{ do(X_i = j) \colon 1 \leq i \leq N \text{ and } j \in \set{0,1}}$
\item For this causal model, $P(Y|do(X_i = j)) = P(Y|X_i = j)$
\item $\vec{q} = (q_1, \ldots, q_N) \in [0,1]^N$ so that $q_i = \P{X_i = 1}$
\item $I_\tau = \set{ i : \min\set{q_i, 1-q_i} < \frac{1}{\tau}}$ is set of 'unbalanced' variables using a threshold $1/\tau$
\item The difficulty of the problem is captured by:
\vspace*{.5cm}
\eq{
\label{eq:m-simple}
m(\vec{q}) = \min \set{ \tau : |I_{\tau}| \leq \tau}\,.
}
\end{itemize} 




\end{block}
\end{column} % End of the first column


%----------------------------------------------------------------------------------------
\begin{column}{1.1\onecolwid} % The 2nd
\begin{block}{Parallel Algorithm}
\setbeamercolor*{block body}{bg=white!95!blue,fg=black}
\begin{algorithm}[H]
\caption{ Parallel Bandit Algorithm \textcolor{white}{\Large A}}\label{alg:simple}
\begin{algorithmic}[1]
\STATE {\bf Input:} Total rounds $T$ and $N$.
\FOR{$t \in 1,\ldots,T / 2$}
\STATE Perform empty intervention $do()$
\STATE Observe $\vec{X}_t$ and $Y_t$
\ENDFOR
\FOR{$a = do(X_i = x) \in \actions$}
\STATE Count times $X_i = x$ seen: $T_a = \sum_{t=1}^{T/2} \ind{X_{t,i} = x}$
\STATE Estimate reward: $\hat{\mu}_a = \frac{1}{T_a} \sum_{t=1}^{T/2} \ind{X_{t,i} = x} Y_t$ \\[0.2cm]
\STATE Estimate probabilities: $\hat{p}_a = \frac{2 T_a}{T}$,\,\, $\hat q_i = \hat p_{do(X_i = 1)}$
\ENDFOR
\STATE Compute $\hat{m} = m(\vec{\hat q})$ and $A = \set{a \in \actions \colon \hat{p}_a \leq \frac{1}{\hat m}}$.
\STATE Let $T_A := \frac{T}{2 |A|}$ be times to sample each $a\in A$.
\FOR{$a = do(X_i = x) \in A$}
\FOR{$t \in 1,\ldots,T_A$}
\STATE Intervene with $a$ and observe $Y_t$
\ENDFOR
\STATE Re-estimate $\hat{\mu}_a = \frac{1}{T_A} \sum_{t=1}^{T_A} Y_t$
\ENDFOR
\RETURN estimated optimal $\hat{a}^*_T \in \argmax_{a\in\actions} \hat{\mu}_a$
\end{algorithmic}
\end{algorithm}

\eq{
&\textbf{upper-bound, } \simpleregret \in \bigo{\sqrt{\frac{m(\vec{q})}{T}\log\left(\frac{NT}{m(\vec{q})}\right)}}\,.\\
&\textbf{lower-bound, }\simpleregret \in \Omega\left(\sqrt{\frac{m(\vec{q})}{T}}\right)\,.\\
&\textbf{standard lower-bound, }\simpleregret \in \Omega\left(\sqrt{\frac{N}{T}}\right)\,.
}
\vspace*{-1cm}
\end{block}

\begin{block}{General Graphs}
\begin{figure}
    \begin{subfigure}[b]{0.48\textwidth}
    \centering    
          \begin{tikzpicture}[->,>=stealth',shorten >=1pt,auto,node distance=1.0em,
  thick,main node/.style={observed}, hidden/.style={empty},background rectangle/.style={fill=olive!45}]
%every node/.style={scale=0.6}
 %nodes
\node[main node](1){$X_{1}$};
\node[main node, right=of 1](2){$X_{2}$};
\node[hidden, right=of 2](3){$...$};
\node[main node, right=of 3](4){$X_{N}$};
\node[main node, below right=of 2](5){$Y$};
\node[main node, above right=of 2](6){$Z$};
 \path[every node/.style={font=\tiny}]
    (1) edge (5)
    	(2) edge (5)
    (4) edge (5)
    (6) edge (1) edge (2) edge (4);
\end{tikzpicture}
        \caption{Confounded parallel graph}
        \label{fig:parallel_confounded} 
    \end{subfigure} 
    \begin{subfigure}[b]{0.48\textwidth}
    \centering    
          \begin{tikzpicture}[->,>=stealth',shorten >=1pt,auto,node distance=1.0em,
  thick,main node/.style={observed}, hidden/.style={empty},background rectangle/.style={fill=olive!45}]
%every node/.style={scale=0.6}
 %nodes
\node[main node](1){$x$};
\node[main node, right=of 1](2){$X_{2}$};
\node[hidden, right=of 2](3){$...$};
\node[main node, right=of 3](4){$X_{N}$};
\node[main node, below right=of 2](5){$Y$};
\node[main node, above right=of 2](6){$Z$};
 \path[every node/.style={font=\tiny}]
    (1) edge (5)
    	(2) edge (5)
    (4) edge (5)
    (6) edge (2) edge (4);
\end{tikzpicture}
        \caption{after $do(X_1=x)$}
        \label{fig:parallel_confounded_post_action} 
    \end{subfigure} 
\caption{Intervention in confounded parallel graph}
\label{fig:causal graphs} 
\end{figure}
\begin{itemize}
\item Known but arbitrary causal graph.
\item It's no longer optimal to split between $do()$ and explicitly playing actions. 
\item We assume the conditional interventional distributions over the parents of $Y$, $\P{\parents{Y}|a}$, are known.
\item $\eta$ is a distribution over interventions from which we sample actions.
\item We use a (truncated) importance weighted estimator to transfer information across different actions.
\eq {
R_a(X,\eta) = \frac{\Pn{a}{\parents{Y}(X)}}{ \sum_{b \in \calA} \eta_b \P{\parents{Y}|b}} \; \leftarrow \text{untruncated weights}
} 
\end{itemize}
\end{block}
\end{column}


\begin{column}{1.1\onecolwid} % The 2nd
\begin{block}{General Algorithm}
\begin{algorithm}[H]
\caption{General Algorithm  \textcolor{white}{\Large A}}\label{alg:general}
\begin{algorithmic}
\STATE {\bf Input:} $T$, $\eta \in [0,1]^{\calA}$, $B \in [0,\infty)^{\calA}$
\FOR{$t \in \set{1,\ldots,T}$}
\STATE Sample action $a_t$ from $\eta$
\STATE Do action $a_t$ and observe $X_t$ and $Y_t$
\ENDFOR
\FOR{$a \in \calA$}
\STATE
\eq {
\hat \mu_a =  \frac{1}{T} \sum_{t=1}^T Y_t R_a(X_t,\eta)  \ind{R_a(X_t,\eta) \leq B_a}
}
\ENDFOR
\STATE {\bf return} $\hat a^*_T = \argmax_a \hat \mu_a$
\end{algorithmic}
\end{algorithm}

\eq{
\textbf{upper-bound, } \simpleregret \in \bigo{\sqrt{\frac{m(\eta)}{T} \log\left(2NT\right)}}\,.
}



\begin{itemize}
\item $\eta$ optimised to minimize the simple regret.
\eq {
\eta^*  = \argmin_\eta \underbrace{\max_{a \in \calA} \EEa \left[R_a(X,\eta)\right]}_{m(\eta)}\,.
} 
\item The difficulty of the problem is captured by $m(\eta*)$.
\begin{itemize}
\item $m(\eta)$ is convex in $\eta$.
\item $m(\eta*) \leq  |\calA|$.
\item In the parallel setting $m(\eta*) \leq 2 m(\boldsymbol{q})$.
\end{itemize}

\end{itemize}


\end{block}
\begin{block}{Experiments}
\begin{figure}
    \begin{subfigure}[t]{0.3\textwidth}
		\centering    
    		\includegraphics[width=\textwidth]{experiment1_20161020_1247.pdf}
    		\caption{Simple regret vs $m(\boldsymbol{q})$. $T=400$, $N = 50$}
        \label{fig:simple_vs_m}
    \end{subfigure}\hfill
    \begin{subfigure}[t]{0.3\textwidth}
    		\centering
        \includegraphics[width=\textwidth]{experiment2_20161020_1249.pdf}
    		\caption{Simple regret vs horizon, $T$, with $N = 50$, $m=2$ and $\epsilon = \sqrt{\frac{N}{8T}}$}
        \label{fig:simple_vs_T_vary_epsilon}
    \end{subfigure}\hfill
    \begin{subfigure}[t]{0.3\textwidth}
    		\centering
    		\includegraphics[width=\textwidth]{experiment3_20161020_1252.pdf}
    		\caption{Simple regret vs horizon, $T$, with $N = 50$, $m=2$ and fixed $\epsilon = .3$}
    		\label{fig:simple_vs_T}
    \end{subfigure}
    \caption{Experimental results on the parallel graph in figure \ref{fig:parallel}}
    \label{fig:experiments}
\end{figure}



\begin{figure}[H]
    \begin{subfigure}[t]{0.3\textwidth}
		\centering    
    		\includegraphics[width=\textwidth]{experiment4_20161023_2120.pdf}
    		\caption{Simple regret vs $m(\eta*)$. $T=400$, $N = 50$}
        \label{fig:simple_vs_m_general}
    \end{subfigure}\hfill
    \begin{subfigure}[t]{0.3\textwidth}
    		\centering
        \includegraphics[width=\textwidth]{experiment7_20161020_1257.pdf}
    		\caption{Simple regret vs horizon, $T$, with $N = 50$ and $m(\eta*)=3.1$ }
        \label{fig:simple_vs_T_general}
    \end{subfigure}\hfill
    \begin{subfigure}[t]{0.3\textwidth}
    		\centering
    		\includegraphics[width=\textwidth]{experiment5_20161023_2118.pdf}
    		\caption{Simple regret vs horizon, $T$, with $N = 21$, $m(\eta*)=4.3$ with no actions setting $Z$}
    		\label{fig:simple_vs_T_misspecified}
    \end{subfigure}
    \caption{Experimental results on the confounded graph in figure \ref{fig:parallel_confounded}}
    \label{fig:experiments_confounded}
\end{figure}

\end{block}

\end{column}

%\begin{column}{\onecolwid} % The 4th column
%
%%----------------------------------------------------------------------------------------
%%	CONCLUSION
%%----------------------------------------------------------------------------------------
%\begin{block}{Conclusion}
%\begin{itemize}
%\item For parallel bandits we have an optimal algorithm (upto log factors) and simple measure of the difficulty of the problem $m(\vec{q})$.
%\item For general graphs, $m(\eta) < N$ measures the difficulty of the problem. Finding an algorithm that only requires the causal graph and lower bounds for its simple regret in the general case is left as future work.
%\end{itemize}
%\end{block}
%
%\begin{block}{Discussion \& Future work}
%
%\begin{itemize}
%\item The parallel bandit problem can be viewed as an instance of a time varying graph feedback problem,
%\begin{itemize}
%\item this does not apply to general causal graphs
%\item algorithms for feedback graphs are not optimal for the parallel bandit.
%\end{itemize}
%
%\item Our algorithms only use the reward signal to estimate $\hat{\mu}_a$, not to inform which arms to sample. Making better use of the reward signal could substantially improve practical performance (but not minimax regret). 
%\item Cumulative Regret
%
%\item Generalize to consider causal bandit problems where the causal graph is partially or completely unknown. 
%
%\end{itemize}
%
%
%
%
%
% 
%
%\end{block}
%
%\begin{block}
%
%%\includegraphics{indiana_university_logo} 
%%\includegraphics[scale=1]{data61_logo.png} 
%%\includegraphics[scale=1]{ANU_logo.jpg} 
%
%
%
%
%\end{block}
%
%%%----------------------------------------------------------------------------------------
%%%	ACKNOWLEDGEMENTS
%%%----------------------------------------------------------------------------------------
%%
%%\setbeamercolor{block title}{fg=red,bg=white} % Change the block title color
%%
%%\begin{block}{Acknowledgements}
%%
%%\small{\rmfamily{Nam mollis tristique neque eu luctus. Suspendisse rutrum congue nisi sed convallis. Aenean id neque dolor. Pellentesque habitant morbi tristique senectus et netus et malesuada fames ac turpis egestas.}} \\
%%
%%\end{block}
%
%%----------------------------------------------------------------------------------------
%%	CONTACT INFORMATION
%%----------------------------------------------------------------------------------------
%
%\setbeamercolor{block alerted title}{fg=black,bg=norange} % Change the alert block title colors
%\setbeamercolor{block alerted body}{fg=black,bg=white} % Change the alert block body colors
%
%%\begin{alertblock}{Contact Information}
%%
%%\begin{itemize}
%%\item Web: \href{http://www.university.edu/smithlab}{http://www.university.edu/smithlab}
%%\item Email: \href{mailto:john@smith.com}{john@smith.com}
%%\item Phone: +1 (000) 111 1111
%%\end{itemize}
%%
%%\end{alertblock}
%
%
%
%%----------------------------------------------------------------------------------------
%
%\end{column} % End of the 4th column

\end{columns} % End of all the columns in the poster

\end{frame} % End of the enclosing frame

\end{document}
