

%%%%%%%%%%%%%%%%%%%%%%%%%%%%%%%%%%%%%%%%%%%%
% PROOF OF SIMPLE REGRET UPPER BOUND
%%%%%%%%%%%%%%%%%%%%%%%%%%%%%%%%%%%%%%%%%%%%
\section{Proof of Theorem \ref{thm:uq-simple}}\label{sec:thm:uq-simple}


Assume without loss of generality that $q_1 \leq q_2 \leq \ldots \leq q_N \leq 1/2$. The assumption is non-restrictive since all variables
are independent and permutations of the variables can be pushed to the reward function.
The proof of Theorem \ref{thm:uq-simple} requires some lemmas, the first of which is an immediate consequence of the Chernoff bound.


\begin{lemma}\label{lem:conc1}
Let $i \in \set{1,\ldots, N}$ and $\delta > 0$. Then
\eq{
\P{\left|\hat q_i - q_i\right| \geq \sqrt{\frac{6q_i}{T} \log \frac{2}{\delta}}} \leq \delta\,.
}
\end{lemma}

\begin{lemma}\label{lem:conc2}
Let $X_1,X_2\ldots,$ be a sequence of random variables with $X_i \in [0,1]$ and $\EE X_i = p$ and $\delta \in [0,1]$.
Then 
\eq{
\P{\exists t \geq n_0 : \left|\frac{1}{t} \sum_{s=1}^t X_s - p\right| \geq \sqrt{\frac{2}{n_0} \log \frac{2}{\delta}}} \leq 4\delta\,.
}
\end{lemma}

\begin{proof}
For $\delta \geq 1/4$ the result is trivial. Otherwise 
by Hoeffding's bound and the union bound:
\eq{
\P{\exists t \geq n_0 : \left|\frac{1}{t} \sum_{s=1}^t X_s - p\right| \geq \sqrt{\frac{2}{n_0} \log \frac{2}{\delta}}} 
&\leq \sum_{t = n_0}^\infty \P{\left|\frac{1}{t} \sum_{s=1}^t X_s - p\right| \geq \sqrt{\frac{2}{n_0} \log \frac{2}{\delta}}} \\
&\leq 2\sum_{t=n_0}^\infty \exp\left(-\frac{t}{n_0} \log \frac{2}{\delta}\right) 
\leq 4\delta\,. \qedhere
}
\end{proof}



\begin{lemma}\label{lem:m_est}
Let $\delta \in (0,1)$ and assume $T \geq 48m \log\frac{2N}{\delta}$. Then
\eq{
\P{2m(\vec{q}) / 3 \leq m(\vec{\hat q}) \leq 2m(\vec{q})} \geq 1 - \delta\,.
}
\end{lemma}

\begin{proof}
Let $F$ be the event that there exists and $1 \leq i \leq N$ for which
\eq{
\left|\hat q_i - q_i\right| \geq \sqrt{\frac{6q_i}{T} \log \frac{2N}{\delta}}\,.
}
Then by the union bound and Lemma \ref{lem:conc1} we have $\P{F} \leq \delta$. The result will be completed by showing that
when $F$ does not hold we have $2m(\vec{q})/3 \leq m(\vec{\hat q}) \leq 2m(\vec{q})$.
From the definition of $m(\vec{q})$ and our assumption on $\vec{q}$ we have for $i > m$ that $q_i \geq q_m \geq 1/m$ and so by Lemma \ref{lem:conc1} we have
\eq{
\frac{3}{4} 
&\geq \frac{1}{2} + \sqrt{\frac{3}{T} \log \frac{2N}{\delta}} 
\geq q_i + \sqrt{\frac{6q_i}{T} \log \frac{2N}{\delta}} 
\geq \hat q_i \\
&\geq q_i - \sqrt{\frac{6q_i}{T} \log \frac{2N}{\delta}}
\geq q_i - \sqrt{\frac{q_i}{8m}}
\geq \frac{1}{2m}\,.
}
Therefore by the pigeonhole principle we have $m(\vec{\hat q}) \leq 2m$.
For the other direction we proceed in a similar fashion. Since the failure event $F$ does not hold we have for $i \leq m$ that
\eq{
\hat q_i 
\leq q_i + \sqrt{\frac{6q_i}{T} \log\frac{2N}{\delta}} 
\leq \frac{1}{m} \left(1 + \sqrt{\frac{1}{8}}\right)
\leq \frac{3}{2m}\,.
}
Therefore $m(\vec{\hat q}) \geq 2m(\vec{q}) / 3$ as required. 
\end{proof}

\begin{proof}[Proof of Theorem \ref{thm:uq-simple}]
Let $\delta = m = m(\vec{q}) / N$. Then by Lemma \ref{lem:m_est} we have 
\eq{
\P{2m/3 \leq m(\vec{\hat q}) \leq 2m} \geq 1 - \delta\,.
}
Recall that $A = \set{a \in \actions : \hat p_a \leq 1/m(\vec{\hat q})}$. Then
for $a \in A$ the algorithm estimates $\mu_a$ from $T/(2m(\vec{\hat q})) \geq T/(4m)$ samples.
Therefore by Hoeffding's inequality and the union bound we have
\eq{
\P{\exists a \in A : |\mu_a - \hat \mu_a| \geq \sqrt{\frac{8m}{T} \log\frac{2N}{\delta}}} \leq \delta\,.
}
For arms not in $a$ we have $\hat p_a \geq 1/m(\vec{\hat q}) \geq 1/(2m)$.
Therefore if $a = do(X_i = j)$, then 
\eq{
\hat p_a = \frac{2}{T} \sum_{t=1}^{T/2} \ind{X_i = j} \geq \frac{1}{2m}\,. 
}
Therefore $\sum_{t=1}^{T/2} \ind{X_{t,i} = j} \geq T/4m$
and by Lemma \ref{lem:conc2} we have
\eq{
\P{\sum_{t=1}^{T/2} \ind{X_i = j} \geq \frac{T}{4m} \text{ and } \left|\hat \mu_a - \mu_a\right| \geq \sqrt{\frac{8m}{T} \log \frac{2N}{\delta}}} \leq 4\delta / N\,.
}
Therefore with probability at least $1 - 6\delta$ we have
\eq{
(\forall a \in \actions) \qquad |\hat \mu_a - \mu_a| \leq \sqrt{\frac{8m}{T} \log \frac{N}{\delta}} = \epsilon\,.
}
If this occurs, then 
\eq{
\mu_{\hat a^*_T} \geq \hat \mu_{\hat a^*_T} - \epsilon \geq \hat \mu_{a^*} - \epsilon \geq \mu_{a^*} - 2\epsilon\,.
}
Therefore
\eq{
\mu^* - \EE[\mu_{\hat a^*_T}] 
\leq 6\delta + \epsilon 
\leq \frac{6m}{T} + \sqrt{\frac{32m}{T} \log \frac{NT}{m}}\,, 
}
which completes the result.
\end{proof}

%%%%%%%%%%%%%%%%%%%%%%%%%%%%%%%%%%%%%%%%%%%%
% LOWER BOUND
%%%%%%%%%%%%%%%%%%%%%%%%%%%%%%%%%%%%%%%%%%%%
\section{Proof of Theorem \ref{thm:lower}}\label{sec:thm:lower}

We follow a relatively standard path by choosing multiple environments that have different optimal arms, but which cannot all be statistically
separated in $T$ rounds.
Assume without loss of generality that $q_1 \leq q_2 \leq \ldots \leq q_N \leq 1/2$.
For each $i$ define reward function $r_i$ by
\eq{
r_0(\boldsymbol{X}) &= \frac{1}{2} &
r_i(\boldsymbol{X}) &= \begin{cases}
\frac{1}{2} + \epsilon & \text{if } X_i = 1 \\
\frac{1}{2} & \text{otherwise}\,,
\end{cases}
}
where $1/4 \geq \epsilon > 0$ is some constant to be chosen later.
We abbreviate $R_{T,i}$ to be the expected simple regret incurred when interacting with the
environment determined by $\boldsymbol{q}$ and $r_i$. Let $\operatorname{P}_i$ be the corresponding measure
on all observations over all $T$ rounds and $\EE_i$ the expectation with respect to $\operatorname{P}_i$. By Lemma 2.6 by \citet{Tsy08} we have
\eq{
\Prz{\hat a^*_T = a^*} + \Pri{\hat a^*_T \neq a^*} \geq \exp\left(-\KL(\operatorname{P}_0, \operatorname{P}_i)\right)\,,
}
where $\KL(\Ps_0, \Ps_i)$ is the KL divergence between measures $\operatorname{P}_0$ and $\operatorname{P}_i$.
Let $T_i(T) = \sum_{t=1}^T \ind{a_t = do(X_i = 1)}$ be the total number of times the learner intervenes on variable $i$ by setting it to $1$.
Then for $i \leq m$ we have $q_i \leq 1/m$ and the KL divergence between $\Ps_0$ and $\Ps_i$ may be bounded using the telescoping property (chain rule) and
by bounding the local KL divergence by the $\chi$-squared distance as by \citet{Auer1995}. This leads to 
\eq{
\KL(\Ps_0, \Ps_i) 
&\leq 6\epsilon^2 \EE_0\left[\sum_{t=1}^T \ind{X_{t,i} = 1}\right] 
\leq 6\epsilon^2 \left(\EE_0 T_i(T) + q_i T\right) 
\leq 6\epsilon^2 \left(\EE_0 T_i(T) + \frac{T}{m}\right)\,.
}
Define set $A = \set{i \leq m : \EE_0 T_i(T) \leq 2T / m}$.
Then for $i \in A$ and choosing $\epsilon = \min\set{1/4, \sqrt{m/(18T)}}$ we have
\eq{
\KL(\Ps_0, \Ps_i) \leq \frac{18T\epsilon^2}{m} = 1\,. 
}
Now $\sum_{i=1}^m \EE_0 T_i(T) \leq T$, which implies that $|A| \geq m/2$.
Therefore
\eq{
\sum_{i \in A} \Pri{\hat a^*_T \neq a} 
\geq \sum_{i \in A} \exp\left(-\KL(\Ps_0, \Ps_i)\right) - 1
\geq \frac{|A|}{e} - 1 
\geq \frac{m}{2e} - 1\,.
}
Therefore there exists an $i \in A$ such that
$\Pri{\hat a^*_T \neq a^*} \geq \frac{\frac{m}{2e} - 1}{m}$. 
Therefore if $\epsilon < 1/4$ we have
\eq{
R_{T,i} \geq \frac{1}{2} \Pn{i}{\hat a^*_T \neq a^*} \epsilon \geq \frac{\frac{m}{2e} - 1}{2m} \sqrt{\frac{m}{18T}}\,.
}
Otherwise $m \geq 18T$ so $\sqrt{m/T} = \Omega(1)$ and
\eq{
R_{T,i} \geq \frac{1}{2} \Pn{i}{\hat a^*_T \neq a^*} \epsilon \geq \frac{1}{4} \frac{\frac{m}{2e} - 1}{2m} \in \Omega(1) 
}
as required.

%%%%%%%%%%%%%%%%%%%%%%%%%%%%%%%%%%%%%%%%%%%%
% GENERAL-GRAPH UPPER BOUND
%%%%%%%%%%%%%%%%%%%%%%%%%%%%%%%%%%%%%%%%%%%%
\section{Proof of Theorem \ref{thm:general}}\label{sec:thm:general}

\begin{proof}
First note that $X_t, Y_t$ are sampled from $\operatorname{Q}$.
We abbreviate $Z_{at} = Z_a(X_t)$ and $R_{at} = R_a(X_t)$.
By definition we have $|Z_{at}| \leq B_a$ and 
\eq{
\Var_Q[Z_{at}] 
\leq \EE_Q[Z_{at}^2] 
\leq \EE_{\Ps_a}\left[\frac{\Ps_a(\parents{Y}(X))}{\Q{\parents{Y}(X)}}\right] 
\leq m(\eta)\,.
}
Checking the expectation we have
\eq{
\EE_Q[Z_{at}] 
= \EE_{\Ps_a}\left[Y \ind{R_{at} \leq B_a}\right] 
= \EE_{\Ps_a} Y - \EE_{\Ps_a} \left[Y\ind{R_a > B_a}\right] 
= \mu_a - \beta_a\,,
}
where 
\eq{
0 \leq \beta_a = \EE_{\Ps_a}[Y \ind{R_a > B_a}] \leq \Pn{a}{R_a(X) > B_a}
}
is the negative bias. 
The bias may be bounded in terms of $m(\eta)$ via an application of Markov's inequality.
\eq{
\beta_a \leq \Pn{a}{R_a(X) > B_a} \leq \frac{\EE_{\Ps_a}[R_a(X)]}{B_a} \leq \frac{m(\eta)}{B_a}\,.
}
Let $\epsilon_a > 0$ be given by
\eq{
\epsilon = \sqrt{\frac{2m(\eta)}{T} \log\left(2T|\calA|\right)} + \frac{3B_a}{T} \log\left(2T|\calA|\right)\,.
}
Then by the union bound and Bernstein's inequality 
\eq{
\P{\text{exists } a \in \calA : \left|\hat \mu_a - \EE_Q[\hat \mu_a]\right| \geq \epsilon_a} 
\leq \sum_{a \in \calA} \P{\left|\hat \mu_a - \EE_Q[\hat \mu_a]\right| \geq \epsilon_a} \leq \frac{1}{T}\,.
}
Assuming this event does not occur and letting $a^* = \argmax_{a \in \calA} \mu_a$ we have
\eq{
\mu_I \geq \hat \mu_I - \epsilon_I  
\geq \hat \mu_{a^*} - \epsilon_I  
\geq \mu^* - \epsilon_{a^*} - \epsilon_I - \beta_{a^*}\,. 
}
By the definition of the truncation
we have
\eq{
\epsilon_a \leq \left(\sqrt{2} + 3\right)\sqrt{\frac{m(\eta)}{T} \log\left(2T|\calA|\right)}
}
and
\eq{
\beta_a \leq \sqrt{\frac{m(\eta)}{T} \log\left(2T|\calA|\right)}\,. 
}
Therefore for $C = \sqrt{2} + 4$ we have
\eq{
\P{\mu_I \geq \mu^* - C \sqrt{\frac{m(\eta)}{T} \log\left(2T|\calA|\right)}} \leq \frac{1}{T}\,.
}
Therefore
\eq{
\mu^* - \EE[\mu_I] \leq C \sqrt{\frac{m(\eta)}{T} \log\left(2T|\calA|\right)} + \frac{1}{T}
}
as required.
\end{proof}

\subsection{Relationship between $m(\eta)$ and $m(\boldsymbol{q})$}

\begin{proposition}\label{pro:m-equivelence} In the parallel bandit setting,
$m(\eta^*) \leq 2m(\boldsymbol{q})$.
\end{proposition} 

\begin{proof}

Let:

\eq{
\eta_a = \begin{cases}
\frac{1}{2} & \text{ for }a=do()\\
\frac{1}{2m(\boldsymbol{q})}\ind{\P{X_i = j} \leq \frac{1}{m(\boldsymbol{q})}} & \text{ for all other }a \in \calA
\end{cases}
}

\eq{
\sum_{a \in \calA}\eta_a = \frac{1}{2}+\sum\ind{\P{X_i = j} \leq \frac{1}{m(\boldsymbol{q})}}
}



\end{proof}
\
%in the parallel bandit setting the $m(\eta)$ given in this section approximately coincides with the $m(\vec{q})$ in Eq.\ \ref{eq:m-simple}.
%Recall in that setting that 
%\eq{
%\actions = \set{do()} \cup \set{do(X_i = j) : 1 \leq i \leq N \text{ and } j \in \set{0,1}}\,.
%}
%For $a = do()$ choose $\eta_a = 1/2$. 
%Let $m = m(\vec{q})$ and note that, by definition, there are at most $m$ pairs $(i,j)$ such that $\P{X_i = j} \leq 1/m$.
%Thus, for $a = do(X_i = j)$ letting $\eta_a \propto \ind{\P{X_i = j} \leq 1/m} / (2m)$ guarantees $\eta_a \ge \frac{1}{2m}$ when $\eta_a \ne 0$.
%It is then easy to check that $m(\eta) \leq 2m$ using an argument like that for Proposition~\ref{pro:m-bound}.


