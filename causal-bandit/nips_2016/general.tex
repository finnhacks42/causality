We now consider the more general problem where the graph structure is known, but arbitrary. For general graphs, $\P{Y|X_i=j} \neq \P{Y|do(X_i=j)}$ (correlation is not causation). However, if all the variables are observable, any causal distribution $\P{X_1...X_N|do(X_i=j)}$ can be expressed in terms of observational distributions via the truncated factorization formula \cite{Pearl2000}. 
\eq{
\P{X_1...X_N|do(X_i=j)} = 
\prod_{k \neq i}\P{X_k|\parents{X_k}}\delta(X_i - j)\,, 
} 
where $\parents{X_k}$ denotes the parents of $X_k$ and $\delta$ is the dirac delta function. 

We could naively generalize our approach for parallel bandits by observing for $T/2$ rounds, applying the truncated product factorization to 
write an expression for each $\P{Y|a}$ in terms of observational quantities and explicitly playing the actions for which the observational 
estimates were poor. However, it is no longer optimal to ignore the information we can learn about the reward for intervening on one variable 
from rounds in which we act on a different variable. Consider the graph in Figure \ref{fig:causalchain} and suppose each variable deterministically 
takes the value of its parent, $X_k = X_{k-1}$ for $k\in {2,\ldots,N}$ and $\P{X_1} = 0$. We can learn the reward for all the interventions $do(X_i = 1)$ 
simultaneously by selecting $do(X_1 = 1)$, but not from $do()$. In addition, variance of the observational estimator for $a = do(X_i = j)$ can be 
high even if $\P{X_i = j}$ is large. Given the causal graph in Figure \ref{fig:causalStructure_confounded}, $\P{Y|do(X_2= j)} = \sum_{X_1}\P{X_1}\P{Y|X_1, X_2 = j}$. 
Suppose $X_2 = X_1$ deterministically, no matter how large $\P{X_2 = 1}$ is we will never observe $(X_2=1,X_1 = 0)$ and so cannot 
get a good estimate for $\P{Y|do(X_2=1)}$. 

To solve the general problem we need an estimator for each action that incorporates information obtained from every other action and a way to optimally 
allocate samples to actions. To address this difficult problem, we assume the conditional interventional distributions $\P{\parents{Y}|a}$ (but not $\P{Y|a}$) 
are known. These could be estimated from experimental data on the same covariates but where the outcome of interest differed, such that $Y$ was not included, 
or similarly from observational data subject to identifiability constraints. Of course this is a somewhat limiting assumption, but seems like a natural place to
start. The challenge of estimating the conditional distributions for all variables in an optimal way is left as an interesting future direction.
Let $\eta$ be a distribution on available interventions $a \in \calA$ so $\eta_a \geq 0$ and $\sum_{a \in \calA} \eta_a = 1$. 
Define $Q = \sum_{a \in \calA} \eta_a \P{\parents{Y}|a}$ to be the mixture distribution over the interventions with respect to $\eta$.






\begin{wrapfigure}[11]{r}{0.5\textwidth}
\vspace{-25pt}
\begin{minipage}{.5\textwidth}
\begin{algorithm}[H]
\caption{General Algorithm}\label{alg:general}
\begin{algorithmic}
\STATE {\bf Input:} $T$, $\eta \in [0,1]^{\calA}$, $B \in [0,\infty)^{\calA}$
\FOR{$t \in \set{1,\ldots,T}$}
\STATE Sample action $a_t$ from $\eta$
\STATE Do action $a_t$ and observe $X_t$ and $Y_t$
\ENDFOR
\FOR{$a \in \calA$}
\STATE
\eq {
\hat \mu_a =  \frac{1}{T} \sum_{t=1}^T Y_t R_a(X_t)  \ind{R_a(X_t) \leq B_a}
}
\ENDFOR
\STATE {\bf return} $\hat a^*_T = \argmax_a \hat \mu_a$
\end{algorithmic}
\end{algorithm}
\end{minipage}
\end{wrapfigure}

Our algorithm samples $T$ actions from $\eta$ and uses them to estimate the returns $\mu_a$ for all $a \in \calA$ simultaneously via a truncated importance weighted estimator. Let $\parents{Y}(X)$ denote the realization of the variables in $X$ that are parents of Y and define $R_a(X) = \frac{\Pn{a}{\parents{Y}(X)}}{\Q{\parents{Y}(X)}}$

\eq {
\hat \mu_a =  \frac{1}{T} \sum_{t=1}^T Y_t R_a(X_t)  \ind{R_a(X_t) \leq B_a}\,, 
} 

where $ B_a \geq 0$  is a constant that tunes the level of truncation to be chosen subsequently. The truncation introduces a bias in the estimator, but simultaneously chops the potentially heavy tail that is so detrimental to its concentration guarantees. 

The distribution over actions, $\eta$ plays the role of allocating samples to actions and is optimized to minimize the worst-case simple regret. Abusing notation we define $m(\eta)$ by
\eq{
m(\eta) = \max_{a \in \calA} \EEa\left[\frac{\Pn{a}{\parents{Y}(X)}}{\Q{\parents{Y}(X)}}\right]\,,\text{ where } \EEa \text{ is the expectation with respect to } \Pn{a}.
}

We will show shortly that $m(\eta)$ is a measure of the difficulty of the problem that approximately coincides with the version for parallel bandits, justifying the name overloading.

\begin{theorem}\label{thm:general}
If Algorithm \ref{alg:general} is run with $B \in \R^{\calA}$ given by $B_a = \sqrt{\frac{m(\eta)T}{\log\left(2T|\calA|\right)}}\,.$

\eq{
\simpleregret \in \bigo{\sqrt{\frac{m(\eta)}{T} \log\left(2T|\calA|\right)}}\,.
}
\end{theorem}
\ifsup 
The proof is in Section \ref{sec:thm:general}.
\else
The proof is in the supplementary materials.
\fi Note the regret has the same form as that obtained for Algorithm \ref{alg:simple}, with $m(\eta)$ replacing $m(q)$. Algorithm \ref{alg:simple} assumes only the graph structure and not knowledge of the conditional distributions on $X$. Thus it has broader applicability to the parallel graph than the generic algorithm given here. We believe that Algorithm \ref{alg:general} with the optimal choice of $\eta$ is close to minimax optimal, but leave lower bounds
for future work.


\paragraph{Choosing the Sampling Distribution} Algorithm \ref{alg:general} depends on a choice of sampling distribution $\operatorname{Q}$ that is determined by $\eta$. In light of Theorem \ref{thm:general}
a natural choice of $\eta$ is the minimiser of $m(\eta)$.
\eq{
\eta^* 
= \argmin_\eta m(\eta) = \argmin_\eta \underbrace{\max_{a \in \calA} \EEa \left[\frac{\Pn{a}{\parents{Y}(X)}}{\sum_{b \in \calA} \eta_b \Pn{b}{\parents{Y}(X)}}\right]}_{m(\eta)}\,.
}
Since the mixture of convex functions is convex and the maximum of a set of convex functions is convex, we see that $m(\eta)$ is convex (in $\eta$).
Therefore the minimisation problem may be tackled using standard techniques from convex optimisation. An interpretation of $m(\eta^*)$ is the minimum achievable worst-case variance of the importance weighted estimator. In the experimental section we present some special cases, but for now we give two simple results. The first shows that $|\calA|$ serves as an upper bound on $m(\eta^*)$.

\begin{proposition}\label{pro:m-bound}
$m(\eta^*) \leq |\calA|$. \textit{Proof.} 
\textup{By definition, $m(\eta^*) \leq m(\eta)$ for all $\eta$. Let $\eta_a = 1/|\calA|\,\forall a$.}
\eq{
m(\eta) 
= \max_a \EEa\left[\frac{\Pn{a}{\parents{Y}(X)}}{\Q{\parents{Y}(X)}}\right] 
\leq \max_a \EEa\left[\frac{\Pn{a}{\parents{Y}(X)}}{\eta_a \Pn{a}{\parents{Y}(X)}}\right] 
= \max_a \EEa\left[\frac{1}{\eta_a}\right] = |\calA| %\qedhere
}
\end{proposition} 

The second observation is that, in the parallel bandit setting, $m(\eta^*) \leq 2m(\boldsymbol{q})$. This is easy to see by letting $\eta_a = 1/2$ for $a = do()$ and $\eta_a = \ind{\P{X_i = j} \leq 1/m(\boldsymbol{q})} / 2m(\boldsymbol{q})$ for the actions corresponding to $do(X_i=j)$, and applying an argument like that for Proposition~\ref{pro:m-bound}. \ifsup 
The proof is in Section \ref{sec:m-equivelence}.
\else
The proof is in the supplementary materials.
\fi

\begin{remark}\label{rem:truncate}
The choice of $B_a$ given in Theorem \ref{thm:general} is not the only possibility. As we shall see in the experiments, it is 
often possible to choose $B_a$ significantly
larger when there is no heavy tail and this can drastically improve performance by eliminating the bias. This is especially true when the ratio $R_a$ is never too large
and Bernstein's inequality could be used directly without the truncation. For another discussion see the article by \citet{BJQ13} who also use importance weighted estimators
to learn from observational data.
\end{remark}






