
\newcommand{\calP}{\mathcal P}
\newcommand{\x}{\boldsymbol{x}}
\newcommand{\Ps}{\operatorname{P}}

We now consider the more general problem where the graph structure is known, but arbitrary. For general graphs, $\P{Y|X_i=j} \neq \P{Y|do(X_i=j)}$ (correlation is not causation). However, if all the variables are observable, any causal distribution $\P{X_1...X_N|do(X_i=j)}$ can be expressed in terms of observational distributions via the truncated factorization formula \cite{Pearl2000}. 
\eq{
\P{X_1...X_N|do(X_i=j)} = 
\prod_{k \neq i}\P{X_k|\parents{X_k}}\delta(X_i - j)  
} 
Where $\parents{X_k}$ denotes the parents of $X_k$ and $\delta$ is the dirac delta function. 

We could naively generalize our approach for parallel bandits by observing for $T/2$ rounds, applying the truncated product factorization to write an expression for each $\P{Y|a}$ in terms of observational quantities and explicitly playing the actions $a$ for which the observational estimates were poor. However, it is no longer optimal to ignore the information we can learn about the reward for intervening on one variable from rounds in which we act on a different variable. Consider the graph in Figure \ref{fig:causalchain}, where each variable deterministically takes the value of its parent, $X_k = X_{k-1}$ 
for $k\in {2,\ldots,N}$ and $\P{X_1} = 0$. 
We can learn the reward for all the interventions $do(X_i = 1)$ simultaneously by selecting $do(X_1 = 1)$, (but not from $do()$). 

To solve the general problem we need an estimator for each action that incorporates information obtained from every other action and a way to optimally allocate samples to actions. NEED TO EXPLAIN WHY THIS PROBLEM IS HARD. 

To address this difficult problem, we firstly note that the causal effect of an action $a$ on $Y$ can be factored into the causal effect of $a$ on the parents of $Y$ and the effect of the parents of $Y$ on $Y$. $\P{Y|a} = \P{Y|\parents{Y}}\P{\parents{Y}|a}$ We then assume the conditional interventional distributions $\P{\parents{Y}|a}$ are known. ADD MORE JUSTIFICATION FOR THIS ASSUMPTION. IE COULD HAPPEN IF WE HAD EXPERIMENTAL DATA COLLECTED WITH A DIFFERENT OBJECTIVE AND IS A NESSESSARY FIRST STEP TO SOLVING THE MORE GENERAL PROBLEM. 


%Firstly, the variance of the observational estimator for $a = do(X_i = j)$ can be high even if $\P{X_i = j}$ is large. For example, given the causal graph in Figure \ref{fig:causalStructure_confounded}, $\P{Y|do(X_2= j)} = \sum_{X_1}\P{X_1}\P{Y|X_1, X_2 = j}$. Suppose $X_2 = X_1$ deterministically, no matter how large $\P{X_2 = 1}$ is we will never observe $(X_2=1,X_1 = 0)$ and so cannot get a good estimate for $\P{Y|do(X_2=1)}$. Secondly it is no longer optimal 

\begin{figure}
    \begin{subfigure}[b]{0.3\textwidth}
        \begin{tikzpicture}[->,>=stealth',shorten >=1pt,auto,node distance=1cm,
  thick,main node/.style={observed}, hidden/.style={empty},background rectangle/.style={fill=olive!45}]
\node[main node](1){$X_1$};
\node[main node, below left=of 1](2){$X_2$};
\node[main node, below right=of 1](4){Y};
 \path[every node/.style={font=\sffamily\small}]
    (1) edge (2)
    (1) edge (4)
    (2) edge (4);
\end{tikzpicture}
        \caption{}
        \label{fig:causalStructure_confounded}
    \end{subfigure}
    \begin{subfigure}[b]{0.69\textwidth}
         \begin{tikzpicture}[->,>=stealth',shorten >=1pt,auto,node distance=1cm,
  thick,main node/.style={observed}, hidden/.style={empty},background rectangle/.style={fill=olive!45}]
\node[main node](1){$X_{1}$};
\node[main node, right=of 1](2){$X_{2}$};
\node[hidden, right=of 2](3){$...$};
\node[main node, right=of 3](4){$X_{N}$};
\node[main node, right=of 4](5){Y};
 \path[every node/.style={font=\sffamily\small}]
    (1) edge (2)
  	(2) edge (3)
    (3) edge (4)
    (4) edge (5);
\end{tikzpicture}
        \caption{}
        \label{fig:causalchain}
    \end{subfigure}
    \caption{Example causal graphs}\label{fig:animals}
\end{figure}

Let $\eta$ be a distribution on available interventions $a \in \calA$ so $\eta_a \geq 0$ and $\sum_{a \in \calA} \eta_a = 1$.
Define $Q = \sum_{a \in \calA} \eta_a \P{\parents{Y}|a}$ to be the mixture distribution over the interventions with respect to $\eta$.

Our algorithm will choose $T$ samples from $Q$ IS THIS LANGUAGE CORRECT. CAN WE SAY IT LIKE THIS? and use them to estimate the returns $\mu_a$ for all $a \in \calA$ simultaneously via a truncated importance weighted estimator. The truncation introduces a bias in the estimator, but simultaneously chops the potentially heavy tail that is so detrimental to its concentration guarantees. 

Define the random variable
\eq{
Z_a(X) = R_a(X) Y \ind{R_a(X) \leq B_a}\,,
}
where $B_a \geq 0$ is some constant to be chosen subsequently and
\eq{
R_a(X) = \frac{\Pn{a}{\parents{Y}(X)}}{\Q{\parents{Y}(X)}}\,.
}
Abusing notation we define $m(\eta)$ by
\eq{
m(\eta) = \max_{a \in \calA} \EE_{\Ps_a}\left[\frac{\Pn{a}{\parents{Y}(X)}}{\Q{\parents{Y}(X)}}\right]\,.
}
We will shortly see that $m(\eta)$ is a measure of the difficulty of the problem that approximately 
coincides with the version for parallel bandits, justifying the name overloading.

\begin{algorithm}[H]
\caption{General Algorithm}\label{alg:general}
\begin{algorithmic}
\STATE {\bf Input:} $T$, $\eta \in [0,1]^{\calA}$, $B \in [0,\infty)^{\calA}$
\FOR{$t \in \set{1,\ldots,T}$}
\STATE Sample action $a_t = a$ with probability $\eta$
\STATE Do action $a_t$ and observe $X_t$ and $Y_t$
\ENDFOR
\STATE For each $a \in \calA$ compute an estimate of its return:
\eq{
\forall a\in \calA \qquad \hat \mu_a = \frac{1}{T} \sum_{t=1}^T Z_a(X_t)
}
\STATE {\bf return} $\hat a^*_T = \argmax_a \hat \mu_a$
\end{algorithmic}
\end{algorithm}

\begin{theorem}\label{thm:general}
If Algorithm \ref{alg:general} is run with $B \in \R^{\calA}$ given by
\eq{
B_a = \sqrt{\frac{m(\eta)T}{\log\left(2T|\calA|\right)}}\,.
}
Then for $C = \sqrt{2} + 4$ we have
\eq{
\mu^* - \EE[\mu_I] \leq C\sqrt{\frac{m(\eta)}{T} \log\left(2T|\calA|\right)} + \frac{1}{T}\,.
}
\end{theorem}




\begin{remark}\label{rem:truncate}
The choice of $B_a$ given in Theorem \ref{thm:general} is not the only possibility. As we shall see in the experiments, it is 
often possible to choose $B_a$ significantly
larger when there is no heavy tail and this can drastically improve performance by eliminating the bias. This is especially true when the ratio $R_a$ is never too large
and Bernstein's inequality could be used directly without the truncation. For another discussion see the article by \citet{BJQ13} who also use importance weighted estimators
to learn from observational data.
\end{remark}

\subsection*{Choosing the Sampling Distribution}

Algorithm \ref{alg:general} depends on a choice of sampling distribution $\operatorname{Q}$ that is determined by $\eta$. In light of Theorem \ref{thm:general}
a natural choice of $\eta$ is the minimiser of $m(\eta)$.
\eq{
\eta^* 
= \argmin_\eta m(\eta) 
&= \argmin_\eta \max_{a \in \calA} \EE_{\Ps_a} \left[\frac{\Pn{a}{\parents{Y}(X)}}{\Q{\parents{Y}(X)}}\right] \\
&= \argmin_\eta \underbrace{\max_{a \in \calA} \EE_{\Ps_a} \left[\frac{\Pn{a}{\parents{Y}(X)}}{\sum_{b \in \calA} \eta_b \Pn{b}{\parents{Y}(X)}}\right]}_{m(\eta)}\,.
}
Since the mixture of convex functions is convex and the maximum of a set of convex functions is convex, we see that $m(\eta)$ is convex (in $\eta$).
Therefore the minimisation problem may be tackled using standard techniques from convex optimisation. 
An interpretation of $m(\eta^*)$ is the minimum achievable worst-case variance of the importance weighted estimator.
In the experimental section we present some special cases,
but for now we give two simple results. The first shows that $|\calA|$ serves as an upper bound on $m(\eta^*)$.

\begin{proposition}\label{pro:m-bound}
$m(\eta^*) \leq |\calA|$.
\end{proposition}

\begin{proof}
By definition, $m(\eta^*) \leq m(\eta)$ for all $\eta$. Let $\eta_a = 1/|\calA|$. Then
\eq{
m(\eta) 
= \max_a \EE_{\Ps_a}\left[\frac{\Pn{a}{\parents{Y}(X)}}{\Q{\parents{Y}(X)}}\right] 
\leq \max_a \EE_{\Ps_a}\left[\frac{\Pn{a}{\parents{Y}(X)}}{\eta_a \Pn{a}{\parents{Y}(X)}}\right] 
\leq \max_a \EE_{\Ps_a}\left[\frac{1}{\eta_a}\right] = |\calA| \qedhere
}
\end{proof}

The second observation is that in the parallel bandit setting the $m(\eta)$ given in this section approximately coincides with the $m(\vec{q})$ in Eq.\ \ref{eq:m-simple}.
Recall in that setting that 
\eq{
\actions = \set{do()} \cup \set{do(X_i = j) : 1 \leq i \leq N \text{ and } j \in \set{0,1}}\,.
}
For $a = do()$ choose $\eta_a = 1/2$. 
Let $m = m(\vec{q})$ and note that, by definition, there are at most $m$ pairs $(i,j)$ such that $\P{X_i = j} \leq 1/m$.
Thus, for $a = do(X_i = j)$ letting $\eta_a \propto \ind{\P{X_i = j} \leq 1/m} / (2m)$ guarantees $\eta_a \ge \frac{1}{2m}$ when $\eta_a \ne 0$.
It is then easy to check that $m(\eta) \leq 2m$ using an argument like that for Proposition~\ref{pro:m-bound}.

\begin{remark}
The algorithm in the previous section for the parallel bandit problem did not assume knowledge of the conditional distributions on $X$ and 
instead only the graph structure. Thus the algorithm given in that setting has broader applicability to that particular graph than the generic algorithm given here.
Finally, we believe that Algorithm \ref{alg:general} with the optimal choice of $\eta$ is close to minimax optimal, but leave lower bounds
for future work.
\end{remark}




