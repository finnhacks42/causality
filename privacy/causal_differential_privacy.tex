\documentclass{article}

%%%%%%%%%%%%%%%%%%%%%%%%%%%%%%%%
% PACKAGES
%%%%%%%%%%%%%%%%%%%%%%%%%%%%%%%%
\usepackage{times}
\usepackage{fullpage}
\usepackage{latexsym}
\usepackage{amsmath}
\usepackage{amssymb}
\usepackage{mathtools}
\usepackage{accents}
\usepackage{tikz}
\usepackage{pgfplots}
\usepackage[ruled]{algorithm}
\usepackage{algpseudocode}
\usepackage{dsfont}
\usepackage[bf]{caption}
\usepackage{hyperref}
\hypersetup{
    bookmarks=true,         % show bookmarks bar?
    unicode=false,          % non-Latin characters in AcrobatÕs bookmarks
    pdftoolbar=true,        % show AcrobatÕs toolbar?
    pdfmenubar=true,        % show AcrobatÕs menu?
    pdffitwindow=false,     % window fit to page when opened
    pdfstartview={FitH},    % fits the width of the page to the window
    pdftitle={My title},    % title
    pdfauthor={Author},     % author
    pdfsubject={Subject},   % subject of the document
    pdfcreator={Creator},   % creator of the document
    pdfproducer={Producer}, % producer of the document
    pdfkeywords={keyword1} {key2} {key3}, % list of keywords
    pdfnewwindow=true,      % links in new window
    colorlinks=true,       % false: boxed links; true: colored links
    linkcolor=red,          % color of internal links (change box color with linkbordercolor)
    citecolor=blue,        % color of links to bibliography
    filecolor=magenta,      % color of file links
    urlcolor=cyan           % color of external links
}
\usepackage{amsthm}
\usepackage{natbib}
\usepackage[capitalize]{cleveref}
\usepackage{graphicx}
\usepackage{parskip}
%%%%%%%%%%%%%%%%%%%%%%%%%%%%%%%%
% MACROS
%%%%%%%%%%%%%%%%%%%%%%%%%%%%%%%%
\newcommand{\defined}{\vcentcolon =}
\newcommand{\rdefined}{=\vcentcolon}
\newcommand{\E}{\mathbb E}
\newcommand{\Var}{\operatorname{Var}}
\newcommand{\calF}{\mathcal F}
\newcommand{\sr}[1]{\stackrel{#1}}
\newcommand{\set}[1]{\left\{#1\right\}}
\newcommand{\ind}[1]{\mathds{1}\!\!\set{#1}}
\newcommand{\argmax}{\operatornamewithlimits{arg\,max}}
\newcommand{\argmin}{\operatornamewithlimits{arg\,min}}
\newcommand{\floor}[1]{\left \lfloor {#1} \right\rfloor}
\newcommand{\ceil}[1]{\left \lceil {#1} \right\rceil}
\newcommand{\eqn}[1]{\begin{align}#1\end{align}}
\newcommand{\eq}[1]{\begin{align*}#1\end{align*}}
\newcommand{\Ber}{\operatorname{Bernoulli}}
\renewcommand{\P}[1]{\operatorname{P}\left\{#1\right\}}


%%%%%%%%%%%%%%%%%%%%%%%%%%%%%%%%
% THEOREMS
%%%%%%%%%%%%%%%%%%%%%%%%%%%%%%%%
\theoremstyle{plain}
\newtheorem{theorem}{Theorem}
\newtheorem{proposition}[theorem]{Proposition}
\newtheorem{lemma}[theorem]{Lemma}
\newtheorem{corollary}[theorem]{Corollary}
\theoremstyle{definition}
\newtheorem{definition}[theorem]{Definition}
\newtheorem{assumption}[theorem]{Assumption}
\newtheorem{remark}[theorem]{Remark}
\newtheorem{example}[theorem]{Example}

\title{Protecting causal effects}
\author{}

\begin{document}
\def\ci{\perp\!\!\!\perp}
\maketitle

Differential privacy has primarily focused on protecting individuals but there is increasing interest in problems relating to hiding certain aggregate properties of a dataset whilst preserving the ability to use it for a specified purpose. In this problem we consider if we can release a dataset for predictive purposes but discourage the inference of causal conclusions about relationships between the covariates. 

Consider a scenario under which the causal graph generating a dataset is considered known - but may contain unmeasured variables. 

Assume we add noise to the dataset via a single point crossover process (see paper). The goal is to prevent reliable estimation of causal effects without effecting our ability to predict a particular target variable. Some specific questions:

\begin{itemize}
\item Under what circumstances (graph structures) it is possible to disrupt inference of a particular causal effect
\item How can we maxiumully disrupt this inference (ie is there a cut that adds more noise for a given amount of shuffling of the data)
\item What about if we want to disrupt a specific set of causal inference questions
\item What about if we want to disrupt as many causal queries as possible.
\end{itemize}

The single point crossover process divides the variables into two groups and makes estimates of the form $P(X_1|X_2)$ unreliable if $X_1$ and $X_2$ are in different partitions. We can think of the process as drawing a cut through the graph.  




\end{document}




