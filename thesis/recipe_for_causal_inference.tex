\documentclass{article}

\usepackage[latin1]{inputenc}
\usepackage{tikz}
\usetikzlibrary{shapes,arrows}
\begin{document}
\pagestyle{empty}


% Define block styles
\tikzstyle{block} = [rectangle, draw, fill=blue!20, 
    text width=10em, text centered, rounded corners, minimum height=3em]
\tikzstyle{line} = [draw, -latex']
    
\begin{tikzpicture}[node distance = 2cm, auto]
    % Place nodes
    \node [block] (drawgraph) {Use prior knowledge to draw the graph};
    \node [block, below of=drawgraph] (docalc) {Apply the Do Calculus};
    \node [block, below left of=docalc,node distance=3.5cm] (fail) {Non Identifiable};
    \node [block, below right of=docalc, node distance=3.5cm] (success) {Identifiable};
    \node [block, below of=fail](notident){Bugger! You will have to make additional assumptions or do experiments};
     \node [block, below of=success](ident){Great! Apply the formula from the do-caclulus to estimate effect.};
     \node [block, fill=red!10, dashed, right of=ident,node distance=4.5cm](stats){Deal with pesky statistical issues introduced by finite data as you see fit};
    
    % Draw edges
    \path [line] (drawgraph) -- (docalc);
    \path [line] (docalc) -- (fail);
    \path [line] (docalc) -- (success);
    \path [line] (fail) -- (notident);
    \path [line] (success) -- (ident);
    \path [line] (stats) -- (ident);
  
\end{tikzpicture}
\end{document}